\documentclass[12pt, titlepage]{article}

\usepackage{amsmath, mathtools}

\usepackage[round]{natbib}
\usepackage{amsfonts}
\usepackage{amssymb}
\usepackage{graphicx}
\usepackage{colortbl}
\usepackage{xr}
\usepackage{hyperref}
\usepackage{longtable}
\usepackage{xfrac}
\usepackage{tabularx}
\usepackage{float}
\usepackage{siunitx}
\usepackage{booktabs}
\usepackage{multirow}
\usepackage[section]{placeins}
\usepackage{caption}
\usepackage{fullpage}

\hypersetup{
bookmarks=true,     % show bookmarks bar?
colorlinks=true,       % false: boxed links; true: colored links
linkcolor=red,          % color of internal links (change box color with linkbordercolor)
citecolor=blue,      % color of links to bibliography
filecolor=magenta,  % color of file links
urlcolor=cyan          % color of external links
}

\usepackage{array}

\externaldocument{../../SRS/SRS}

%%% Comments

\usepackage{color}

\newif\ifcomments\commentstrue %displays comments
%\newif\ifcomments\commentsfalse %so that comments do not display

\ifcomments
\newcommand{\authornote}[3]{\textcolor{#1}{[#3 ---#2]}}
\newcommand{\todo}[1]{\textcolor{red}{[TODO: #1]}}
\else
\newcommand{\authornote}[3]{}
\newcommand{\todo}[1]{}
\fi

\newcommand{\wss}[1]{\authornote{blue}{SS}{#1}} 
\newcommand{\plt}[1]{\authornote{magenta}{TPLT}{#1}} %For explanation of the template
\newcommand{\an}[1]{\authornote{cyan}{Author}{#1}}

%%% Common Parts

\newcommand{\progname}{ProgName} % PUT YOUR PROGRAM NAME HERE
\newcommand{\authname}{Team \#, Team Name
\\ Student 1 name
\\ Student 2 name
\\ Student 3 name
\\ Student 4 name} % AUTHOR NAMES                  

\usepackage{hyperref}
    \hypersetup{colorlinks=true, linkcolor=blue, citecolor=blue, filecolor=blue,
                urlcolor=blue, unicode=false}
    \urlstyle{same}
                                


\begin{document}

\title{Module Interface Specification for CRLP (Concrete Remaining Life Perdiction)}

\author{Yi-Leng Chen}

\date{March 18, 2024}

\maketitle

\pagenumbering{roman}

\section{Revision History}

\begin{tabularx}{\textwidth}{p{3cm}p{2cm}X}
\toprule {\bf Date} & {\bf Version} & {\bf Notes}\\
\midrule
March 18 & 1.0 & Initial Documentation\\
\bottomrule
\end{tabularx}

~\newpage

\section{Symbols, Abbreviations and Acronyms}

See SRS Documentation at \href{https://github.com/kypss94132/CAS741_YiLeng-Chen/blob/main/docs/SRS/SRS.pdf}{GitHub}

\newpage

\tableofcontents

\newpage

\pagenumbering{arabic}

\section{Introduction}

The following document details the Module Interface Specifications for CRLP (Concrete Remaining Life Perdiction)

Complementary documents include the \href{https://github.com/kypss94132/CAS741_YiLeng-Chen/blob/main/docs/SRS/SRS.pdf}{System Requirement Specifications} and \href{https://github.com/kypss94132/CAS741_YiLeng-Chen/blob/main/docs/Design/SoftArchitecture/MG.pdf}{Module Guide }. The full documentation and implementation can be found at \href{https://github.com/kypss94132/CAS741_YiLeng-Chen/tree/main/docs}{GitHub repository}

\section{Notation}

The structure of the MIS for modules comes from \citet{HoffmanAndStrooper1995}, with the addition that template modules have been adapted from \cite{GhezziEtAl2003}. The mathematical notation comes from Chapter 3 of \citet{HoffmanAndStrooper1995}. For instance, the symbol := is used for a multiple assignment statement and conditional rules follow the form $(c_1 \Rightarrow r_1 | c_2 \Rightarrow r_2 | ... | c_n \Rightarrow r_n )$.

The following table summarizes the primitive data types used by CRLP.

\begin{center}
\renewcommand{\arraystretch}{1.2}
\noindent 
\begin{tabular}{l l p{7.5cm}} 
\toprule 
\textbf{Data Type} & \textbf{Notation} & \textbf{Description}\\ 
\midrule
character & char & a single symbol or digit\\
integer & $\mathbb{Z}$ & a number without a fractional component in (-$\infty$, $\infty$) \\
natural number & $\mathbb{N}$ & a number without a fractional component in [1, $\infty$) \\
real & $\mathbb{R}$ & any number in (-$\infty$, $\infty$)\\
\bottomrule
\end{tabular} 
\end{center}

\noindent
The specification of CRLP uses some derived data types: sequences, strings, and tuples. Sequences are lists filled with elements of the same data type. Strings are sequences of characters. Tuples contain a list of values, potentially of
different types. In addition, CRLP uses functions, which
are defined by the data types of their inputs and outputs. Local functions are described by giving their type signature followed by their specification.

\section{Module Decomposition}

The following table is taken directly from the Module Guide document for this project.

\begin{table}[h!]
\centering
\begin{tabular}{p{0.3\textwidth} p{0.6\textwidth}}
\toprule
\textbf{Level 1} & \textbf{Level 2}\\
\midrule

{Hardware-Hiding Module} & ~ \\
\midrule

\multirow{5}{0.3\textwidth}{Behaviour-Hiding Module} 
& Input Parameters Module\\
& Input Format Module\\
& Output Format Module\\
& Control Module\\
& Prediction Calculate Equations Module\\
\midrule

\multirow{1}{0.3\textwidth}{Software Decision Module} 
& Sequence Data Structure Module\\
\bottomrule

\end{tabular}
\caption{Module Hierarchy}
\label{TblMH}
\end{table}

\newpage
~\newpage
\section{MIS of Control Module}
The module provides the main program and GUI to call other modules.

\subsection{Module}
main

\subsection{Uses}
\begin{itemize}
    \item Input Parameters Module
    \item Input Format Module
    \item Output Format Module
    \item Prediction Calculate Equations Module
\end{itemize}

\subsection{Syntax}

\subsubsection{Exported Constants}

\subsubsection{Exported Access Programs}

\begin{center}
\begin{tabular}{p{2cm} p{4cm} p{4cm} p{2cm}}
\hline
\textbf{Name} & \textbf{In} & \textbf{Out} & \textbf{Exceptions} \\
\hline
main & - & - & - \\
\hline
\end{tabular}
\end{center}

\subsection{Semantics}

\subsubsection{State Variables}

None

\subsubsection{Environment Variables}

None

\subsubsection{Assumptions}

None

\subsubsection{Access Routine Semantics}

main():
\begin{itemize}
\item transition: Interact with the GUI and control other modules by following these steps:
    - The necessary data is input by conducting the Input Parameters Module (M2).\\
    - The Input Format Module (M3) converts the input data into the data structure.\\
    - Once the verified data is obtained, the Prediction Calculate Equations Module (M6) begins the calculation process.\\
    - The result is then returned by the Output Format Module (M4).\\

\item output: None
\item exception: The exceptions are listed in corresponding submodules.

\end{itemize}
\subsubsection{Local Functions}
None
\newpage
%----------------------------------------------------%
\section{MIS of Input Format Module}
The module converts the input data into the data structure utilized by the input parameters module.

\subsection{Module}
format\_input

\subsection{Uses}
\begin{itemize}
    \item Input Parameters Module
\end{itemize}

\subsection{Syntax}

\subsubsection{Exported Constants}

\subsubsection{Exported Access Programs}

\begin{center}
\begin{tabular}{p{5cm} p{2cm} p{2cm} p{5cm}}
\hline
\textbf{Name} & \textbf{In} & \textbf{Out} & \textbf{Exceptions} \\
\hline
input\_SourceVerify & - & boolean & input\_DataSourceError\\
input\_Verify & String, $\mathbb{R}$ & $\mathbb{R}$ & input\_FormatError\\
\hline
\end{tabular}
\end{center}

\subsection{Semantics}

\subsubsection{State Variables}

IsSourceExist: Boolean \# Ensure that the necessary columns of data exist on the \href{https://climate.weather.gc.ca/historical_data/search_historic_data_e.html} {Government of Canada} website.

\subsubsection{Environment Variables}

Data source: When conducting the prediction process using weather data, the script retrieves data from the \href{https://climate.weather.gc.ca/historical_data/search_historic_data_e.html} {Government of Canada} website.

\subsubsection{Assumptions}

None

\subsubsection{Access Routine Semantics}

input\_Verify():
\begin{itemize}
\item transition: The following steps are used to load, verify, and store the input data:
- 

\item output: None
\item exception: The exceptions are listed in corresponding submodules.

\end{itemize}
\subsubsection{Local Functions}
None
\newpage

%----------------------------------------------------%
\section{MIS of Input Parameters Module}
The module stores the parameters necessary for CRLP, including concrete properties and weather conditions.

\subsection{Module}
parameters

\subsection{Uses}
\begin{itemize}
    \item Hardware-Hiding Module
\end{itemize}

\subsection{Syntax}

\subsubsection{Exported Constants}
None
\subsubsection{Exported Access Programs}

\begin{center}
\begin{tabular}{p{2cm} p{4cm} p{4cm} p{2cm}}
\hline
\textbf{Name} & \textbf{In} & \textbf{Out} & \textbf{Exceptions} \\
\hline
parameters & - & - & - \\
\hline
\end{tabular}
\end{center}

\subsection{Semantics}
The parameters data structure is utilized to store the input data passed by the Input Format Module.
\subsubsection{State Variables}
The following theories are retrieved from section 4.1.1 of the VnVPlan document \cite{VNV2024}, which aims to identify the symbols used in each equation.

\begin{enumerate}
    \item Theory 1 \\
    $C_{const}$ (constant chloride concentration at the concrete surface): $\mathbb{N}$ \\
    $C_x$ (chloride concentration at depth $x$): $\mathbb{R}$ \\
    $D_\text{cl}$ (chloride ion diffusion coefficient): $\mathbb{R}$\\
    $x$ (depth): $\mathbb{R}$\\
    
    \item Theory 2 \\
    $L$ (remaining uncarbonated cover): $\mathbb{R}$ \\
    $R_c$ (rate of carbonation): $\mathbb{R}$ \\
    
    \item Theory 3 \\
    $k_a$ (coefficient of active corrosion): $\mathbb{R}$ \\
    $k_c$ (quality coefficient of concrete): $\mathbb{R}$ \\
    $k_e$ (coefficient of environment): $\mathbb{R}$\\
    $C_R$ (thickness of concrete cover): $\mathbb{R}$\\
    
    \item Theory 4 \\
    $M_{70}$ (months that relative humidity below 70\%): $\mathbb{Z}$\\
    $M_{100}$ (months that relative humidity between 70\% and 100 \%): $\mathbb{Z}$\\
    $M_{rain}$ (months that rain occurs): $\mathbb{Z}$\\

    \item Theory 5 \\
    $A_d$ (cumulative depth of deterioration after $t_{use}$ years): $\mathbb{N}$\\
    $A_\text{df}$ (concrete cover thickness after $t_{use}$ years): $\mathbb{N}$\\
    $k_d$ (constant encapsulates various factors affecting the degradation of concrete): $\mathbb{N}$\\
    $n$ (time order): $\mathbb{R}$\\
\end{enumerate}

\subsubsection{Environment Variables}
Keyboard: The hardware component used for input. This module assumes the presence of a keyboard for data entry.

\subsubsection{Assumptions}

None

\subsubsection{Access Routine Semantics}

parameters():
\begin{itemize}
\item transition: This module contains a data structure for storing the input values formatted by the Input Format Module.
\item output: None
\item exception: None
\end{itemize}

\subsubsection{Local Functions}
None
\newpage

\bibliographystyle {plainnat}
\bibliography {MIS.bib}

\newpage

\section{Appendix} \label{Appendix}

\end{document}