\documentclass[12pt, titlepage]{article}
\usepackage{array} 
\usepackage{booktabs}
\usepackage{adjustbox}
\usepackage{tabularx}
\newcounter{rowcount}
\setcounter{rowcount}{0}
\usepackage{hyperref}
\hypersetup{
    colorlinks,
    citecolor=blue,
    filecolor=black,
    linkcolor=red,
    urlcolor=blue
}
\usepackage[round]{natbib}

%%% Comments

\usepackage{color}

\newif\ifcomments\commentstrue %displays comments
%\newif\ifcomments\commentsfalse %so that comments do not display

\ifcomments
\newcommand{\authornote}[3]{\textcolor{#1}{[#3 ---#2]}}
\newcommand{\todo}[1]{\textcolor{red}{[TODO: #1]}}
\else
\newcommand{\authornote}[3]{}
\newcommand{\todo}[1]{}
\fi

\newcommand{\wss}[1]{\authornote{blue}{SS}{#1}} 
\newcommand{\plt}[1]{\authornote{magenta}{TPLT}{#1}} %For explanation of the template
\newcommand{\an}[1]{\authornote{cyan}{Author}{#1}}

%%% Common Parts

\newcommand{\progname}{ProgName} % PUT YOUR PROGRAM NAME HERE
\newcommand{\authname}{Team \#, Team Name
\\ Student 1 name
\\ Student 2 name
\\ Student 3 name
\\ Student 4 name} % AUTHOR NAMES                  

\usepackage{hyperref}
    \hypersetup{colorlinks=true, linkcolor=blue, citecolor=blue, filecolor=blue,
                urlcolor=blue, unicode=false}
    \urlstyle{same}
                                


\begin{document}

\title{Project Title: System Verification and Validation Plan for Concrete Remaining Life Prediction} 
\author{Yi-Leng Chen}
\date{February 19, 2024}
	
\maketitle

\pagenumbering{roman}

\section*{Revision History}

\begin{tabularx}{\textwidth}{p{3cm}p{2cm}X}
\toprule {\bf Date} & {\bf Version} & {\bf Notes}\\
\midrule
February 19 & 1.0 & Initial Documentation\\
March 4 & 1.1 & Changes based on feedback \\

\bottomrule
\end{tabularx}

\newpage

\tableofcontents

\listoftables

\newpage

\section{Symbols, Abbreviations, and Acronyms}

Table \ref{tab:symbols} lists the symbols, abbreviations, and acronyms used in this document. For other symbols used in this project, they are organized in the first section of the SRS document \cite{srs}.
\newline

\renewcommand{\arraystretch}{1.2}
\begin{tabular}{l l} 
  \toprule		
  \textbf{symbol} & \textbf{description}\\
  \midrule 
  CI & Continuous Integration \\
  CRLP & Concrete Remaining Life Prediction program \\
  MIS & Module Interface Specification \\
  MG & Module Guide \\
  SRS & Software Requirements Specification \\
  VnV & Verification \& Validation \\
  UI & User Interface\\
  \bottomrule
\end{tabular}\\
\label{tab:symbols}

\newpage

\pagenumbering{arabic}

This document provides an overview of the Verification and Validation (VnV) process for the Concrete Remaining Life Prediction program (CRLP), ensuring adherence to the program's specifications outlined in the SRS document \cite{srs}. It begins with general information about CRLP in section \ref{sec:plan}. The subsequent sections will discuss the verification plan and the process of system tests.

\section{General Information}

\subsection{Summary}

This document reviews the VnV process for the CRLP program. By inputting weather data or the condition of the concrete, CRLP can predict the remaining service life of concrete structures, thereby helping to prevent hazards to buildings.

\subsection{Objectives}

The objectives of CRLP will primarily focus on accuracy and usability. In other words, the priority of this program is to ensure that the output is correct and that users can operate the program effortlessly. The program will utilize certain Python external libraries, which we assume to be functioning correctly in this case.

\subsection{Relevant Documentation}

The relevant documentation for CRLP comprises the Problem Statement, which delineates the proposed idea; the System Requirements Specifications, which furnish information about the requirements of the proposed system; the VnV Report, dedicated to validation and verification; and the MG and MIS design documents.

\section{Plan}\label{sec:plan}

The following section outlines the testing plan for CRLP. It begins with addressing the verification and validation team in section \ref{sec:vnvteam}. Subsequently, these sections discuss various plans, such as the SRS verification plan, design verification plan, and implementation verification plan. Section \ref{sec:autoTnV} describes automated testing and verification. Finally, section \ref{sec:softwarevp} mentions the Software validation plan.

\subsection{Verification and Validation Team} \label{sec:vnvteam}

\begin{table}
    \centering
    \begin{tabular}{|c|p{13cm}|} 
    \hline
        Name & Responsibility  \\
    \hline
        Dr.Spencer Smith & As the instructor of the team, the role is to supervise and provide professional assistance to the author.\\
    \hline
        Yi-Leng Chen & As the author of the team, responsibilities include composing documents, implementing the program, and testing it.\\
    \hline
        Atiyeh Sayadi & As the Domain Expert of the team, responsibilities include reviewing all documents related to the project.  \\
    \hline
        Waqar Awan & As the Secondary Reviewer of the team, responsibilities include reviewing the VnV plan.\\
    \hline
        Kim Ying Wong & As the Secondary Reviewer of the team, responsibilities include reviewing MG and MIS documents. \\
    \hline
    \end{tabular}
    \caption{Verification and Validation Team}
    \label{tab:VnVTeam}
\end{table}

\subsection{SRS Verification Plan}
The SRS document can be verified by following these steps:
\begin{enumerate}
    \item Documentation: The document should adhere to the SRS template \cite{srstem} and be uploaded to GitHub.
    \item Review: The author create separate issues for the instructor, domain expert, and secondary reviewer (specifically designated for SRS).
    These three individuals are responsible for reviewing the SRS document using the SRS checklist\cite{srscheck}, one by one.
    \item Modify: Make adjustments to the document based on the feedback received from the reviewers. Then, repeat the aforementioned steps.
    
\end{enumerate}

\subsection{Design Verification Plan}
The design documents, such as the Module Guide (MG) and the Module Interface Specification (MIS), can be verified by following these steps:
\begin{enumerate}
    \item Documentation: Unlike previous documents, there are no templates for design documents. The MG should provide a high-level overview of the architecture and structure of the software system, while the MIS should provide interface details about the modules. These documents need to be uploaded to GitHub once they are finished.
    \item Review: The author create separate issues for the instructor, domain expert, and secondary reviewer (specifically designated for specific review). These three individuals are responsible for using checklist \cite{mgcheck} \cite{mischeck} to review the documents, one by one.
    \item Modify: Make adjustments to the document based on the feedback received from the reviewers. Then, repeat the aforementioned steps.
\end{enumerate}

\subsection{Verification and Validation Plan}

The VnV plan can be verified by following these steps:

\begin{enumerate}
    \item Documentation: There is also no restricted format for the document, but there's a template \cite{vnvtem} that can be referred to, and the document needs to be uploaded to GitHub once it is finished.
    \item Review: Once the VnV plan is done, the author create separate issues for the instructor, domain expert, and secondary reviewer (specifically designated for VnV). These three individuals are responsible for using VnV checklist \cite{vnvcheck}to review the document, one by one.
    \item Modify: Make adjustments to the document based on the feedback received from the reviewers. Then, repeat the aforementioned steps.
\end{enumerate}

\subsection{Implementation Verification Plan}

This project will only conduct dynamic testing. The testing plans for functional and non-functional requirements are mentioned in section \ref{sec:systests}, while unit tests are listed in section \ref{sce:unittests}.

\subsection{Automated Testing and Verification Tools} \label{sec:autoTnV}

\begin{itemize}
    \item Pytest: The test framework for Python, can execute test cases and provide test results.
    \item Flake8: A Python linter tool that standardizes the style and quality of Python code.
    \item GitHub Actions: A Continuous Integration (CI) tool which can automate the execution of tests whenever changes are pushed to a repository.
\end{itemize}

\subsection{Software Validation Plan} \label{sec:softwarevp}

Due to the lack of an external supervisor for this project and time constraints, a software validation plan will not be conducted.

\section{System Test Description} \label{sec:systests}
	
\subsection{Tests for Functional Requirements}
In Section 5.1 of the SRS document \cite{srs}, four functional requirements were listed. R1 and R2 primarily emphasize the input and output functions. R3 involves testing the calculated function, while R4 verifies its results.

\subsubsection{Input function test}
The core of this project involves prediction using various theories, each of which has different equations. The test plans are designed based on these theories.

\begin{enumerate}
    \item Theory 1: Chloride concentration and depth: 
    \label{item:browne}
    R.D. Browne \cite{browne} forecasted the remaining service life by using the chloride concentration and its depth.

    \item Theory 2: Carbonation cover:
    \label{item:hookham}
    C.J. Hookham\cite{hookham} forecasted the remaining service life using the time to full cover carbonation.
    
    \item Theory 3: Chloride attack:
    \label{item:hookham}
    C.J. Hookham\cite{hookham} forecasted the remaining service life using using chloride attack level.
    
    \item Theory 4: Weather data:
    \label{item:weather}
     C. Andrade et al \cite{andrade} used humidity data from weather records to predict the remaining service life of concrete.

    \item Theory 5: Degradation factors:
    \label{item:failure}
    J.R. Clifton \cite{clifton} forecasted the remaining service life using various degradation factors.
\end{enumerate}

\begin{table}
    \centering
    \begin{adjustbox}{width=\textwidth}
    \begin{tabularx}{\textwidth}{|X|c|c|c|} % Adjusting for three columns
    \hline
       Parameters & \multicolumn{3}{|c|}{Input} \\
    \hline
        \multicolumn{4}{|c|}{Theory 1}\\
    \hline
        Constant chloride concentration ($C_0$) & 1000 & None& -1000\\
    \hline
         Chloride ion diffusion coefficient ($D_{cl}$) & 1x10^{-10} & None &1x10^{-10}\\
    \hline
        Chloride depth (x) & 0.01 & None & 0.01\\
    \hline
        Chloride concentration & 899.8 & None & 899.8\\
    \hline
        Output & Success & Error & Error\\
    \hline
        \multicolumn{4}{|c|}{Theory 2} \\
    \hline
        Remaining Uncarbonated Cover & 0.95 & None & 0.95\\
    \hline
        Chloride concentration & 0.028 & None & -0.028\\
    \hline
        Output & Success & Error &Error\\
    \hline
        \multicolumn{4}{|c|}{Theory 3} \\
    \hline
        Quality coefficient of concrete($K_c$) & 7.59 & None & 7.59\\
    \hline
        Coefficient of environment($K_e$) & 0.028 & None&0.028\\
    \hline
         Thickness of concrete cover($K_R$) & 0.95 & None &-0.95\\
    \hline
         Coefficient of active corrosion($K_a$) & 4 & None &4\\
    \hline
        Output & Success & Error &Error \\
    \hline
        \multicolumn{4}{|c|}{Theory 4} \\ 
    \hline
        Months that relative humidity is below 70\%($M_1$) & 4 & None & 1\\
     \hline
        Months that relative humidity is between 70\% and 100\%($M_2$) & 6 & None & 1\\
    \hline
        Months that rain occurs($M_3)& 2 & None & 1\\
    \hline
        Output & Success & Error & Error\\
    \hline
        \multicolumn{4}{|c|}{Theory 5} \\ 
    \hline
       Degradation Category & Carbonation & None & Carbonation\\
    \hline
       Years of Concrete Usage & 36 & None & -1\\
    \hline
        Average Depth of Carbonation& 24 & None & 24\\
    \hline
        Amount of damage at failure& 48 & None & 48\\
    \hline
        Output & Success & Error & Error \\
    \hline
    \multicolumn{4}{p{\dimexpr\textwidth-2\tabcolsep\relax}}{*Success means CRLP output the predict result successfully.} \\
    \multicolumn{4}{p{\dimexpr\textwidth-2\tabcolsep\relax}}{*Error means rejecting input and popping error message: Please enter the value!} \\
    \end{tabularx}
    \end{adjustbox}
    \caption{input data test}
    \label{tab:inputtest}
\end{table}

\newpage
\subsubsection{Output function test}

\begin{table}
    \centering
    \begin{adjustbox}{width=\textwidth}
    \begin{tabularx}{\textwidth}{|c|c|c|X|} % Corrected to use X for the last column
    \hline
        Case ID & Input & Output & Theory\\
    \hline
        1 & Input random valid value of data & Success & All theories\\
    \hline
        2 & Input nothing for input data & Nothing & All theories\\
    \hline
    \multicolumn{4}{p{\dimexpr\textwidth-2\tabcolsep\relax}}{*Success means CRLP output the predict result successfully.} \\
    \end{tabularx}
    \end{adjustbox}
    \caption{Output data test}
    \label{tab:outputtest}
\end{table}

    \item test-output-functions \\
    Control: Automatic	\\				
    Initial State: Uninitialized\\		
    Input: Listed in Input column in the table \ref{tab:outputtest}.\\			
    Output: Listed in Input column in the table \ref{tab:outputtest}.\\
    Test Case Derivation: This case aims to test the different behaviors after inputting data and conducting calculations.\\
    How test will be performed: This test will be conducted through Pytest and GitHub Actions.\\
					
\end{enumerate}

\subsubsection{Calculated function test}

\begin{table}
    \centering
    \begin{adjustbox}{width=\textwidth}
    \begin{tabularx}{\textwidth}{|c|c|c|c|} % Use "X" for the last column to allow it to stretch to fill the remaining space
    \hline
        Case ID & Input & Output & Theory\\
    \hline
        1 & Input random valid value of data & Success & All theories\\
    \hline
        2& Input nothing for input data & Nothing & All theories\\
    \hline
    \multicolumn{4}{p{\dimexpr\textwidth-2\tabcolsep\relax}}{*Success means CRLP starting to calculate result.} \\ 
    \end{tabularx}
    \end{adjustbox}
    \caption{calculated data test}
    \label{tab:caltest}
\end{table}

\begin{enumerate}

    \item test-output-functions \\
    Control: Automatic	\\				
    Initial State: Uninitialized\\		
    Input: Listed in Input column in the table \ref{tab:caltest}.\\			
    Output: Listed in Input column in the table \ref{tab:caltest}.\\
    Test Case Derivation: This case aims to test calculate process.\\
    How test will be performed: This test will be conducted through Pytest and GitHub Actions.\\
					
\end{enumerate}

\subsubsection{Result check test}

\begin{table}
    \centering
    \begin{adjustbox}{width=\textwidth}
    \begin{tabularx}{\textwidth}{|c|c|c|c|} % Use "X" for the last column to allow it to stretch to fill the remaining space
    \hline
        Case ID & Input & Output & Theory\\
    \hline
        1 & Input random valid value of data & Success & All theories\\
    \hline
    \multicolumn{4}{p{\dimexpr\textwidth-2\tabcolsep\relax}}{*Success is defined as the accurate calculation of results by CRLP.} \\ 
    \end{tabularx}
    \end{adjustbox}
    \caption{result test}
    \label{tab:resulttest}
\end{table}

\begin{enumerate}

    \item test-output-functions \\
    Control: Manual	\\				
    Initial State: Uninitialized\\		
    Input: Listed in Input column in the table \ref{tab:resulttest}.\\			
    Output: Listed in Input column in the table \ref{tab:resulttest}.\\
    Test Case Derivation: The objective of this case is to verify the accuracy and conformity of the calculated results with the specified equations.\\
    How test will be performed: This test will involve comparing the calculated results with manually calculated values.\\
					
\end{enumerate}
\subsection{Tests for Nonfunctional Requirements}

In Section 5.2 of the SRS document \cite{srs}, three functional requirements were listed.

\subsubsection{Nonfunctional Requirements Test}
		
\paragraph{Usability Test}

\begin{enumerate}

\item Test Usability \\
Type: Manual \\
Initial State: None \\
Input/Condition: Survey for users \\
Output/Result: The survey result from users \\
How test will be performed: Administering the survey (based on appendix section) to users. Following practical usage of the system's UI, sending the survey to users to gather feedback from them.

\item{test-accuracy\\}

    Control: Manual	\\				
    Initial State: Uninitialized\\		
    Input: Listed in Input column in the table \ref{tab:resulttest}.\\			
    Output: Listed in Input column in the table \ref{tab:resulttest}.\\
    Test Case Derivation: The objective of this case is to verify the accuracy and conformity of the calculated results with the specified equations.\\
    How test will be performed: This test will involve comparing the calculated results with manually calculated values.\\

\item{test-reusable\\}

    Control: Manual	\\				
    Initial State: Uninitialized\\		
    Input: elements in each equations.\\			
    Output: generate different predict result\\
    Test Case Derivation: The objective of this case is to ensure that this program can flexibly extend its functions.\\
    How test will be performed: This test will change some coefficients of the current equation to prove that the program can maintain a state-of-the-art view.\\
\end{enumerate}

\subsection{Traceability Between Test Cases and Requirements}

\begin{table}
    \centering
    \begin{tabular}{|c|c|c|c|c|c|c|c|}
    \hline 
         & R1 & R2 & R3 & R4 & NFR1 & NFR2 & NFR3\\
    \hline 
        4.1.1 & X &  &  &  &  &  &\\
    \hline 
        4.1.2 &  & X &  &  &  &  &\\
    \hline 
        4.1.3 &  &  & X &  &  &  &\\
    \hline 
        4.1.4 &  &  &  & X &  &  &\\
    \hline 
        4.2.1 &  &  &  &  & X &  &\\
    \hline 
        4.2.2 &  &  &  &  &  & X &\\
    \hline 
        4.2.3 &  &  &  &  &  &  & X\\
    \hline 
    \end{tabular}
    \caption{Traceability between test Cases and requirements}
    \label{tab:reference table}
\end{table}

\newpage
\section{Unit Test Description} \label{sce:unittests}

\subsection{Unit Testing Scope}
This section will not be completed until after the MIS and MG are finished.

\subsection{Tests for Functional Requirements}
This section will not be completed until after the MIS and MG are finished.

\subsubsection{Module 1}
The following content will not be completed until after the MIS and MG are finished.

\begin{enumerate}

\item{test-id1\\}

Type: TBD \\
Initial State: TBD	\\			
Input: TBD	\\			
Output: TBD \\
Test Case Derivation: TBD \\
How test will be performed: TBD \\
    
\end{enumerate}

\subsection{Tests for Nonfunctional Requirements}
This section will not be completed until after the MIS and MG are finished.

\subsubsection{Module 1}
The following content will not be completed until after the MIS and MG are finished.		
\begin{enumerate}

\item{test-id1\\}

Type: TBD \\				
Initial State: TBD	\\			
Input/Condition: TBD \\				
Output/Result: TBD	\\			
How test will be performed: TBD \\

\end{enumerate}
				
\begin{thebibliography}{9} 
    \bibitem{srs}
    Yi-Leng Chen. (2024). \textit{Software Requirements Specification for: Concrete Remaining Life Prediction.}    \url{https://github.com/kypss94132/CAS741_YiLeng-Chen/blob/main/docs/SRS/SRS.pdf}

    \bibitem{srstem}
    Smith, W. Spencer.(2024). \textit{SRS}. GitHub. \url{https://github.com/smiths/capTemplate/blob/main/docs/SRS/SRS.pdf}
    
    \bibitem{vnvtem}
    Smith, W. Spencer.(2023). \textit{VnVPlan}. GitHub. \url{https://github.com/smiths/capTemplate/blob/main/docs/VnVPlan/VnVPlan.pdf}

    \bibitem{srscheck}
    Smith, W. Spencer.(2024). \textit{SRS-Checklist}. GitHub. \url{https://github.com/kypss94132/CAS741_YiLeng-Chen/blob/main/docs/Checklists/SRS-Checklist.pdf}

    \bibitem{vnvcheck}
    Smith, W. Spencer.(2022). \textit{VnV-Checklist}. GitHub. \url{https://github.com/kypss94132/CAS741_YiLeng-Chen/blob/main/docs/Checklists/VnV-Checklist.pdf}

    \bibitem{mgcheck}
    Smith, W. Spencer.(2022). \textit{MG-Checklist}. GitHub. \url{https://github.com/smiths/capTemplate/blob/main/docs/Checklists/MG-Checklist.pdf}

    \bibitem{mischeck}
    Smith, W. Spencer.(2022). \textit{MIS-Checklist}. GitHub. \url{https://github.com/smiths/capTemplate/blob/main/docs/Checklists/MIS-Checklist.pdf}

    \bibitem{browne}
    R.D. Browne. \textit{Mechanisms of Corrosion of Steel in Concrete}. in Concrete in Relation to Design, Inspection, and Repair of Offshore and Coastal Structures, ACI SP-65.

    \bibitem{hookham}
    C.J. Hookham, \textit{Rehabilitation of Great Lakes Steel’s No. One Dock} ACI Symposium of Durability of Concrete, in press.

    \bibitem{andrade}
    C. Andrade, C. Alonso, and J.A. Gonzalez.(1989). \textit{Approach to the Calculation of the Residual Life in Corroding Concrete Reinforcement based on Corrosion Intensity Values}.9th European Congress on Corrosion: Life Time Expectancy of Materials and Constructions, Vol. 2, Ultrect, Netherlands.

    \bibitem{clifton}
    J.R. Clifton. (1991) \textit{Predicting the Remaining Service Life of Concrete} NISTIR 4712, National Institute of Standards and Technology.
    
\end{thebibliography}
\newpage

\section{Appendix}

\subsection{Usability Survey Questions}
\label{sec:survey}
The following questions comprise the survey, which will be conducted through Google Forms. Users can rate their feelings on a scale from 1 to 5, with 1 representing "extremely disagree" and 5 representing "extremely agree".

\begin{enumerate}
    \item The interface is easy to understand and operate.
    \item How satisfied are you with the speed and efficiency of our experts in conducting math calculations?
    \item The program is equipped with useful tools to aid in research.
    \item The program provides the results I want to know.
    \item Does the program meet your expectations in terms of accuracy and reliability?
    \item How would you rate the overall quality of CRLP?
    \item Do you have any suggestions or feedback for improving our website or services? (An input section will be provided for this question.)
\end{enumerate}

\end{document}