% THIS DOCUMENT IS TAILORED TO REQUIREMENTS FOR SCIENTIFIC COMPUTING.  IT SHOULDN'T
% BE USED FOR NON-SCIENTIFIC COMPUTING PROJECTS
\documentclass[12pt]{article}

\usepackage{amsmath, mathtools}
\usepackage{amsfonts}
\usepackage{amssymb}
\usepackage{graphicx}
\usepackage{colortbl}
\usepackage{xr}
\usepackage{hyperref}
\usepackage{longtable}
\usepackage{xfrac}
\usepackage{tabularx}
\usepackage{float}
\usepackage{siunitx}
\usepackage{booktabs}
\usepackage{caption}
\usepackage{pdflscape}
\usepackage{afterpage}

\usepackage[round]{natbib}

%\usepackage{refcheck}

\hypersetup{
    bookmarks=true,        % show bookmarks bar?
    colorlinks=true,       % false: boxed links; true: colored links
    linkcolor=red,          % color of internal links (change box color with linkbordercolor)
    citecolor=green,        % color of links to bibliography
    filecolor=magenta,      % color of file links
    urlcolor=cyan           % color of external links
}

%%% Comments

\usepackage{color}

\newif\ifcomments\commentstrue %displays comments
%\newif\ifcomments\commentsfalse %so that comments do not display

\ifcomments
\newcommand{\authornote}[3]{\textcolor{#1}{[#3 ---#2]}}
\newcommand{\todo}[1]{\textcolor{red}{[TODO: #1]}}
\else
\newcommand{\authornote}[3]{}
\newcommand{\todo}[1]{}
\fi

\newcommand{\wss}[1]{\authornote{blue}{SS}{#1}} 
\newcommand{\plt}[1]{\authornote{magenta}{TPLT}{#1}} %For explanation of the template
\newcommand{\an}[1]{\authornote{cyan}{Author}{#1}}

%%% Common Parts

\newcommand{\progname}{ProgName} % PUT YOUR PROGRAM NAME HERE
\newcommand{\authname}{Team \#, Team Name
\\ Student 1 name
\\ Student 2 name
\\ Student 3 name
\\ Student 4 name} % AUTHOR NAMES                  

\usepackage{hyperref}
    \hypersetup{colorlinks=true, linkcolor=blue, citecolor=blue, filecolor=blue,
                urlcolor=blue, unicode=false}
    \urlstyle{same}
                                


% For easy change of table widths
\newcommand{\colZwidth}{1.0\textwidth}
\newcommand{\colAwidth}{0.13\textwidth}
\newcommand{\colBwidth}{0.82\textwidth}
\newcommand{\colCwidth}{0.1\textwidth}
\newcommand{\colDwidth}{0.05\textwidth}
\newcommand{\colEwidth}{0.8\textwidth}
\newcommand{\colFwidth}{0.17\textwidth}
\newcommand{\colGwidth}{0.5\textwidth}
\newcommand{\colHwidth}{0.28\textwidth}

% Used so that cross-references have a meaningful prefix
\newcounter{defnum} %Definition Number
\newcommand{\dthedefnum}{GD\thedefnum}
\newcommand{\dref}[1]{GD\ref{#1}}
\newcounter{datadefnum} %Datadefinition Number
\newcommand{\ddthedatadefnum}{DD\thedatadefnum}
\newcommand{\ddref}[1]{DD\ref{#1}}
\newcounter{theorynum} %Theory Number
\newcommand{\tthetheorynum}{TM\thetheorynum}
\newcommand{\tref}[1]{TM\ref{#1}}
\newcounter{tablenum} %Table Number
\newcommand{\tbthetablenum}{TB\thetablenum}
\newcommand{\tbref}[1]{TB\ref{#1}}
\newcounter{assumpnum} %Assumption Number
\newcommand{\atheassumpnum}{A\theassumpnum}
\newcommand{\aref}[1]{A\ref{#1}}
\newcounter{goalnum} %Goal Number
\newcommand{\gthegoalnum}{GS\thegoalnum}
\newcommand{\gsref}[1]{GS\ref{#1}}
\newcounter{instnum} %Instance Number
\newcommand{\itheinstnum}{IM\theinstnum}
\newcommand{\iref}[1]{IM\ref{#1}}
\newcounter{reqnum} %Requirement Number
\newcommand{\rthereqnum}{R\thereqnum}
\newcommand{\rref}[1]{R\ref{#1}}
\newcounter{nfrnum} %NFR Number
\newcommand{\rthenfrnum}{NFR\thenfrnum}
\newcommand{\nfrref}[1]{NFR\ref{#1}}
\newcounter{lcnum} %Likely change number
\newcommand{\lthelcnum}{LC\thelcnum}
\newcommand{\lcref}[1]{LC\ref{#1}}

\usepackage{fullpage}

\newcommand{\deftheory}[9][Not Applicable]
{
\newpage
\noindent \rule{\textwidth}{0.5mm}

\paragraph{RefName: } \textbf{#2} \phantomsection 
\label{#2}

\paragraph{Label:} #3

\noindent \rule{\textwidth}{0.5mm}

\paragraph{Equation:}

#4

\paragraph{Description:}

#5

\paragraph{Notes:}

#6

\paragraph{Source:}

#7

\paragraph{Ref.\ By:}

#8

\paragraph{Preconditions for \hyperref[#2]{#2}:}
\label{#2_precond}

#9

\paragraph{Derivation for \hyperref[#2]{#2}:}
\label{#2_deriv}

#1

\noindent \rule{\textwidth}{0.5mm}

}

\begin{document}

\title{Software Requirements Specification for CRLP: Concrete Remaining Life Prediction} 
\author{Yi-Leng Chen}
\date{February 26, 2024}
	
\maketitle
%~\newpage

\pagenumbering{roman}

\tableofcontents

~\newpage

\section*{Revision History}

\begin{tabularx}{\textwidth}{p{3cm}p{2cm}X}
\toprule {\bf Date} & {\bf Version} & {\bf Notes}\\
\midrule
February 5 & 1.0 & Initial Documentation\\
February 26 & 2.0 & Updates Following Feedback\\
\bottomrule
\end{tabularx}

~\newpage

\section{Reference Material}

This section records information for easy reference.

\subsection{Table of Units}

Throughout this document SI (Syst\`{e}me International d'Unit\'{e}s) is employed as the unit system.  In addition to the basic units, several derived units are used as described below.  For each unit, the symbol is given followed by a description of the unit and the SI name.
~\newline

\renewcommand{\arraystretch}{1.2}
%\begin{table}[ht]
  \noindent \begin{tabular}{l l l} 
    \toprule		
    \textbf{symbol} & \textbf{unit} & \textbf{SI}\\
    \midrule 
    \si{\ampere} & electric current & ampere\\
    \si{\centi\meter} & length & centimeter \\
    \si{\liter} & volume & liter \\
    \si{\metre} & length & metre\\
    \si{\milli\gram} & mass & milligram \\
    \si{\milli\metre} & length & millimeter\\
    \si{\second} & time & second\\
    \text{y} & time & year\\
    

    \bottomrule
  \end{tabular}
  %	\caption{Provide a caption}
%\end{table}

\subsection{Table of Symbols}

The table that follows summarizes the symbols used in this document along with
their units. The symbols are listed in alphabetical order.

\renewcommand{\arraystretch}{1.2}
%\noindent \begin{tabularx}{1.0\textwidth}{l l X}
\noindent \begin{longtable*}{l l p{12cm}} \toprule
\textbf{symbol} & \textbf{unit} & \textbf{description}\\
\midrule 
$A_\text{cr}$ & \si[per-mode=symbol]{{}\micro\ampere\per\centi\meter\squared}& average corrosion rate during one year\\
$A_d$ & \si[per-mode=symbol] {\milli\meter} & cumulative depth of deterioration after $t_{use}$ years\\
$A_\text{df}$ & \si[per-mode=symbol] {\milli\meter} & concrete cover thickness after $t_{use}$ years\\%跟下面要不融合在一起

$C_x$ & \si{\milli\gram}/\si{\liter} & chloride concentration at depth $x$ \\ 
$C_{const}$ & \si{\milli\gram}/\si{\liter} & constant chloride concentration at the concrete surface\\
$C_R$ & cm & thickness of concrete cover\\

$D_\text{cl}$ & $m^2$/s & chloride ion diffusion coefficient\\
$\math{erf}$ & - & error function\\

$H_r$ & \% & relative humidity\\

$k_a$ & - & coefficient of active corrosion \\
$k_c$ & - & quality coefficient of concrete \\
$k_d$ & \si[per-mode=symbol]{{}\milli\meter\per\text{y}} & constant encapsulates various factors affecting the degradation of concrete \\
$k_e$ & - & coefficient of environment \\

$L$ & cm & remaining uncarbonated cover \\

$n$ & - & time order\\

$R_c$ & cm/y & rate of carbonation \\
$R_d$ & \si[per-mode=symbol]{{}\meter\per\second} & Rate of concrete degradation \\

$t$ & y & time required to reach a sufficient depth $x$ to obtain complete chloride concentration\\
$t_{cover}$ & y & time to be fully covered by carbonation\\
$t_{remain}$ & y & remaining service life of concrete\\
$t_{use}$ & y & Years of concrete usage\\
$t_y$ & y & total lifespan of concrete\\

$x$ & m & depth\\
\\ 
\bottomrule
\end{longtable*}

\subsection{Abbreviations and Acronyms}

\renewcommand{\arraystretch}{1.2}
\begin{tabular}{l l} 
  \toprule		
  \textbf{symbol} & \textbf{description}\\
  \midrule 
  A & Assumption\\
  CRLP & Concrete remaining life prediction \\
  DD & Data Definition\\  
  GD & General Definition\\
  GS & Goal Statement\\
  IM & Instance Model\\
  LC & Likely Change\\
  PS & Physical System Description\\
  R & Requirement\\
  SRS & Software Requirements Specification\\
  TM & Theoretical Model\\
  UI & User Interface \\
  \bottomrule
\end{tabular}\\


\pagenumbering{arabic}

\section{Introduction}
In response to the critical need for predicting the remaining life of concrete structures, this project aims to develop a program that implements theories introduced by a United States laboratory in 1992. \cite{concrete_life_1992} Their work emphasized the importance of estimating remaining service life to enable property owners to plan for repairs or demolitions proactively. By predicting the remaining service life of concrete structures, it offers a valuable tool for effective maintenance and decision-making. \\
\\
The upcoming section will present a roadmap for the Software Requirements Specification (SRS) of CRLP. This segment elucidates the document's purpose, outlines the scope of the requirements, and describes the characteristics of the target audience for this document.

\subsection{Purpose of Document}

The primary aim of this document is to outline the requirements of the CRLP program. This encompasses background information, goals, assumptions, theoretical models, definitions, and other details related to model derivation. These components collectively enable the audience to gain a clear understanding and verify the purpose and scientific basis of this program.

\subsection{Scope of Requirements} 
The scope of requirements assumes that parameters from different equations will not influence each other. In other words, if an equation only requires two specific parameters, $L$ and $R_{c}$ to calculate time, the result will not be affected by relative humidity from other equations.

\subsection{Characteristics of Intended Reader} \label{sec_IntendedReader}
The intended reviewers of this documentation should possess a background in meteorology or have completed university-level courses related to concrete construction or materials introduction. Additionally, a basic grasp of high-school level mathematics and chemistry is recommended. For users of the prediction program can have a lower level of expertise, as further explained in the ``User Characteristics'' section (Section~\ref{SecUserCharacteristics}).

\subsection{Organization of Document}

The organization of this document follows the SRS template \cite{srstem} provided by Dr. Smith and the GlassBR SRS document \cite{glassbr_spec}.

\section{General System Description}

This section provides general information about the system.  It identifies the interfaces between the system and its environment, describes the user characteristics and lists the system constraints.  

\subsection{System Context}
{Figure \ref{fig:syscon}} illustrates the system context. In this representation, a circle signifies an external entity, which, in this case, is the user. The rectangle represents the prediction program, and arrows depict the data flow between the system and its environment.

\begin{figure}
    \centering
    \includegraphics[width=0.9\linewidth]{System Context.JPG}
    \caption{System Context}
    \label{fig:syscon}
\end{figure}

The interaction between the product and the user is through a UI. The responsibilities of the user and the system are as follows: 

\begin{itemize}
\item User Responsibilities: Provide the input data related to climate and concrete deterioration level.
\end{itemize}
\begin{itemize}
\item Prediction Program Responsibilities: \\
    - Determine if the inputs satisfy the required program constraints.\\
    - Predict the remaining service life of concrete. \\
\end{itemize}

\subsection{User Characteristics} \label{SecUserCharacteristics}
\begin{itemize}
\item The end user is expected to have a fundamental level of computer literacy in order to effectively navigate the software interface and input concrete details accurately. This includes basic skills such as opening the software application and correctly entering concrete parameters into the system.
\item The end user is expected to have completed university-level civil engineering courses related to concrete introduction or concrete construction. This background knowledge is necessary to comprehend the theoretical aspects of concrete degradation measurement and its underlying causes.
\end{itemize}

\subsection{System Constraints}

There are no system constraints.

\section{Specific System Description}

This section first presents the problem description, which gives a high-level view of the problem to be solved.  This is followed by the solution characteristics specification, which presents the assumptions, theories, definitions and finally
the instance models. 

\subsection{Problem Description} \label{Sec_pd}

A system is required to predict the remaining service life of concrete by integrating climate data and concrete degradation levels.

\subsubsection{Terminology and  Definitions}

This subsection provides a list of terms that are used in the subsequent sections and their meaning, with the purpose of reducing ambiguity and making it easier to correctly understand the requirements:

\begin{enumerate}
    \item Corrosion rates (Icorr): As states on Corrosionpedia \cite{corrosionpedia} website (2019), “Corrosion rate is the speed at which any metal in a specific environment deteriorates. It also can be defined as the amount of corrosion loss per year in thickness.”

    \item Deterioration Factors:
    \begin{itemize}
        \item Carbonation: “It reduces its alkalinity sufficiently to depassivate the steel and initiate corrosion. Carbonation involves the reaction of gaseous carbon dioxide with calcium hydroxide of concrete to form calcium carbonate. The steel's pH becomes susceptible to corrosion when fully carbonated.” (J.R. Clifton, 1991, p.19)\cite{concrete_life_1991}
    \end{itemize}
    \begin{itemize}
        \item Diffusion of chloride ions: It refers to the process by which chloride ions permeate or spread through concrete structures. Once present in concrete, chloride ions can initiate corrosion of steel reinforcement. The diffusion of chloride ions through concrete is influenced by factors such as the presence of entrained air, the spacing of air bubbles, and the pore structure of the concrete matrix. (J.R. Clifton, 1991, p.6)\cite{concrete_life_1991}
    \end{itemize}
    \begin{itemize}
        \item Acid attack (siliceous aggregate): It occurs as a result of alkali-silica reaction (ASR). ASR is a type of alkali-aggregate reaction where the reactive components, namely silica in the aggregate, interact with alkalis in the cement. This reaction leads to the formation of expansive products, causing serious cracking in concrete due to the imbibition of water by the reaction products. (J.R. Clifton, 1991, p.15)\cite{concrete_life_1991}
    \end{itemize}
    \begin{itemize}
        \item Acid attack (carbonate aggregate): It involves alkali-carbonate reaction (ACR). Similar to alkali-silica reaction, alkali-carbonate reaction occurs between alkalis in the cement and certain carbonate aggregates, such as dolomitic limestone aggregates. This reaction can also lead to the formation of expansive products and subsequent cracking of concrete.(J.R. Clifton, 1991, p.15)\cite{concrete_life_1991}
    \end{itemize}
    \begin{itemize}
        \item Sulfate attack: As J.R. Clifton (1991) \cite{concrete_life_1991} states, “Sulfate attack of concrete can result in cracking or disintegration of the material. Naturally occurring sulfates of sodium, potassium, calcium, and magnesium are commonly found in groundwater and soil, particularly in areas with high clay content.”(p.25)
    \end{itemize}
    \begin{itemize}
        \item Frost attack: This is the damage incurred by concrete when it is exposed to cycles of freezing and thawing. Frost damage can occur even if the concrete is not fully saturated with water, as the critical level of saturation for most concretes is around 85 percent. (J.R. Clifton, 1991, p.18)\cite{concrete_life_1991}
    \end{itemize}
    \begin{itemize}
        \item Active corrosion of steel: It refers to the phase in which corrosion is actively occurring, leading to the deterioration of the reinforcing steel within concrete structures.(J.R. Clifton, 1991, p.40)\cite{concrete_life_1991}
    \end{itemize}
    \begin{itemize}
        \item Reinforcement (propagation): It refers to the process of corrosion spreading or advancing within the steel reinforcement embedded in concrete structures. It encompasses the progression of corrosion beyond the initial stage of initiation and extends to the broader understanding of how corrosion develops and spreads over time.(Andrade. C, 2018) \cite{propagation}
    \end{itemize}

    \item Rain: Precipitation in the form of liquid water droplets greater than 0.5 mm. 

    \item Relative Humidity: Relative humidity in percent (\%) is the ratio of the quantity of water vapour the air contains compared to the maximum amount it can hold at that particular temperature. 
\end{enumerate}

\subsubsection{Physical System Description} \label{sec_phySystDescrip}

The physical system regarding how deterioration occurs is not discussed in this study.

\subsubsection{Goal Statements}

Develop a prediction program that implements the theories, with the goal of predicting the remaining service life of concrete structures.

\subsection{Solution Characteristics Specification}

The instance models that govern this program are presented in
Subsection~\ref{sec_instance}.  The information to understand the meaning of the
instance models and their derivation is also presented, so that the instance
models can be verified.

\subsubsection{Assumptions} \label{sec_assumpt}

This section simplifies the original problem and helps in developing the theoretical model by filling in the missing information for the physical system. The numbers given in the square brackets refer to the theoretical model [TM],
general definition [GD], data definition [DD], instance model [IM], or likely change [LC], in which the respective assumption is used.

\begin{itemize}

\item[A\refstepcounter{assumpnum}\theassumpnum \label{kdconstant}:]
  The values of $k_d$ remain constant throughout the entire deterioration period.

\item[A\refstepcounter{assumpnum}\theassumpnum \label{factors}:]
  The factors influencing the remaining life of concrete and causing degradation are restricted to the following: carbonation, diffusion of chloride ions, acid attack (siliceous aggregate), acid attack (carbonate aggregate), sulfate attack, frost attack, active corrosion of steel and reinforcement (propagation).

\item[A\refstepcounter{assumpnum}\theassumpnum \label{kd}:]
    If only one degradation process occurs, implicitly refer to \hyperref[tab:factors]{Figure \ref{tab:factors}} to determine the exact value of n; Multiple degradation processes occurring simultaneously, making the time order n=1.

\item[A\refstepcounter{assumpnum}\theassumpnum 
\label{Rh}:]
When using weather data to predict results, relative humidity is considered the only factor that influences Icorr.

\item[A\refstepcounter{assumpnum}\theassumpnum 
\label{Months}:]
The relative humidity value needs to remain consistently below 70\% for a month to be considered in the calculation of the $M_1$ value. The same applies to the $M_2$ and $M_3$ values. 

\item[A\refstepcounter{assumpnum}\theassumpnum 
\label{Relationship}:]
The relationship between $M_1$ and $M_2$ is mutually exclusive to $M_3$. In other words, there is no rain during the $M_1$ and $M_2$ periods.

\end{itemize}

\subsubsection{Theoretical Models}\label{sec_theoretical}

There are no theoretical models.

\subsubsection{General Definitions}\label{sec_gendef}

There are no general definitions.

\subsubsection{Data Definitions}\label{sec_datadef}

This section collects and defines all the data needed to build the instance models. The dimension of each quantity is also given.  
~\newline

\noindent
\begin{minipage}{\textwidth}
\renewcommand*{\arraystretch}{1.5}
\begin{tabular}{| p{\colAwidth} | p{\colBwidth}|}
\hline
\rowcolor[gray]{0.9}
Number& DD\refstepcounter{datadefnum}\thedatadefnum \label{MC}\\
\hline
Label & Coefficient weighing the impact of corrosion rates (Icorr) in $M$ \\
\hline
Symbol & $M_1$, $M_2$, $M_3$, $C_1$, $C_2$, $C_3$ \\
\hline
SI Units & - \\
\hline
Equation & 
\begin{cases}
    M_1, C_1 = 0.1 \\
    M_2, C_2 = 1.0 \\
    M_3, C_3 = 10 \\
\end{cases} \\
\hline
Description & 
The coefficient for $M_1$ will be 0.1, for $M_2$ it will be 1.0, and for $M_3$ it will be 10.\\
\hline
  Sources& James Clifton et al.~\cite{glassbr_spec} \\
  \hline
  Ref.\ By & \iref{ewat}\\
  \hline
\end{tabular}
\end{minipage}\\

\noindent
\begin{minipage}{\textwidth}
\renewcommand*{\arraystretch}{1.5}
\begin{tabular}{| p{\colAwidth} | p{\colBwidth}|}
\hline
\rowcolor[gray]{0.9}
Number& DD\refstepcounter{datadefnum}\thedatadefnum \label{Acr}\\
\hline
Label & Average corrosion rate of year\\
\hline
Symbol & $A_\text{cr}$ \\
\hline
SI Units & - \\
\hline
Equation & \(A_\text{cr} = \frac{C_1 \cdot M_1 + C_2 \cdot M_2 + C_3 \cdot M_3}{12}\) \\
\hline
Description 
& $M_1$ is month Number of months that relative humidity is below 70\%  \\
& $M_2$ is month Number of months that relative humidity is between 70 and 100\% \\
& $M_3$ is month Number of months that rain occurs\\
& $C_1$ is coefficient weighing the impact of corrosion rates (Icorr) in $M_1$\\
& $C_2$ is coefficient weighing the impact of corrosion rates (Icorr) in $M_2$ \\
& $C_1$ is coefficient weighing the impact of corrosion rates (Icorr) in $M_3$\\
\hline
  Sources& James Clifton et al.~\cite{glassbr_spec} \\
  \hline
  Ref.\ By & \iref{ewat}\\
  \hline
\end{tabular}
\end{minipage}\\
~\newline
%---------------------------------------------------------
\noindent
\begin{minipage}{\textwidth}
\renewcommand*{\arraystretch}{1.5}
\begin{tabular}{| p{\colAwidth} | p{\colBwidth}|}
\hline
\rowcolor[gray]{0.9}
Number& DD\refstepcounter{datadefnum}\thedatadefnum \label{kd}\\
\hline
Label & Amount of concrete degradation\\
\hline
Symbol & $k_d$, $A_d$, $t_y$, n \\
\hline
SI Units & s \\
\hline
Equation & 
\(k_d = \left(\frac{A_d}{t_y}\right)^n\) \\
\hline
Description & 
  $A_d$ is Accumulated deterioration at time $t_y$. \\
  &$k_d$ is Factors influencing deterioration. \\
  &$t_y$ is Service life of concrete. \\
  & n is time order \\
  & The above equation calculate the amount of concrete degradation.\\
\hline
  Sources& James Clifton et al.~\cite{glassbr_spec} \\
  \hline
  Ref.\ By & \iref{ewat}\\
  \hline
\end{tabular}
\end{minipage}\\
~\newline
%---------------------------------------------------------
\noindent
\begin{minipage}{\textwidth}
\renewcommand*{\arraystretch}{1.5}
\begin{tabular}{| p{\colAwidth} | p{\colBwidth}|}
\hline
\rowcolor[gray]{0.9}
Number& DD\refstepcounter{datadefnum}\thedatadefnum \label{ty}\\
\hline
  Label& \bf Time to failure\\
  \hline
Symbol & $t_y$, $A_\text{df}$, $k_d$ , n \\
\hline
SI Units & s \\
\hline
Equation & 
$t_y$ = \left(\frac{{A_d_f}}{k_d}\right)^{\frac{1}{n}}) \\
\hline
Description & 
  \(t_y\) is Service life of concrete \\
  & \(A_d\) is the amount of damage at failure \\
  & \(k_d\) is the factor influencing deterioration \\
  & \(n\) is the time order \\
  & The above equation calculates the time to failure \\
\hline
  Sources& James Clifton et al.~\cite{glassbr_spec} \\
  \hline
  Ref.\ By & \iref{ewat}\\
  \hline
\end{tabular}
\end{minipage}\\
~\newline


\subsubsection{Instance Models} \label{sec_instance}    

This section transforms the problem defined in Section~\ref{Sec_pd} into 
one which is expressed in mathematical terms. It uses concrete symbols defined 
in Section~\ref{sec_datadef} to replace the abstract symbols in the models 
identified in Sections~\ref{sec_theoretical} and~\ref{sec_gendef}.

~\newline

%Instance Model 1

\noindent
\begin{minipage}{\textwidth}
\renewcommand*{\arraystretch}{1.5}
\begin{tabular}{| p{\colAwidth} | p{\colBwidth}|}
  \hline
  \rowcolor[gray]{0.9}
  Number& IM\refstepcounter{instnum}\theinstnum \\
  \hline
  \label{cxy}\\ % Move \label command to a separate line
  Label& \bf By using chloride concentration at depth x to derive time t\\
  \hline
  Input&$C_o$, x, $D_\text{cl}, t$\\
  \hline
  Output& t\\
  \hline
  Description&
  $C_o$ is constant chloride concentration at the surface.(mg/L)\\
  &x is the depth of chloride concentration.(m)\\
  &$D_\text{cl}$ is Chloride ion diffusion coefficient.($m^2/s$)\\
  &t is the time required to reach a sufficient depth x to obtain complete chloride concentration.(year)\\
  & The time t to reach depth x can be used to derive the remaining service life.
  \\
  \hline
  Sources& James Clifton et al.~\cite{glassbr_spec} \\
  \hline
  Ref.\ By & -\\
  \hline
\end{tabular}
\end{minipage}\\
~\newline


%Instance Model 2

\noindent
\begin{minipage}{\textwidth}
\renewcommand*{\arraystretch}{1.5}
\begin{tabular}{| p{\colAwidth} | p{\colBwidth}|}
  \hline
  \rowcolor[gray]{0.9}
  Number& IM\refstepcounter{instnum}\theinstnum \label{tc}\\
  \hline
  Label& \bf Predict remaining service life of carbonation model\\
  \hline
  Input& L, x, $R_c$\\
  \hline
  Output & $t_c$\\
  \hline
  Description&
  $t_c$ is the time to be fully covered by carbonation.(year)\\
  &L is the the remaining uncarbonated cover.(cm)\\
  &$R_c$ is the rate of carbonation.(cm/year)\\
  & The above equation predict remaining service life of carbonation model.
  \\
  \hline
  Sources& James Clifton et al.~\cite{glassbr_spec} \\
  \hline
  Ref.\ By & -\\
  \hline
\end{tabular}
\end{minipage}\\
~\newline

%Instance Model 3

\noindent
\begin{minipage}{\textwidth}
\renewcommand*{\arraystretch}{1.5}
\begin{tabular}{| p{\colAwidth} | p{\colBwidth}|}
  \hline
  \rowcolor[gray]{0.9}
  Number& IM\refstepcounter{instnum}\theinstnum \label{t2}\\
  \hline
  Label& \bf Prediction of remaining service life with respect to chloride attack\\
  \hline
  Input& $k_c$, $k_e$, CR, $k_a$ \\
  \hline
  Output & $t_r$\\
  \hline
  Description&
  $t_r$ is the remaining service life of concrete.(year)\\
  &$k_c$ is the quality coefficient of concrete.(unitless)\\
  &$k_e$ is the coefficient of environment.(unitless)\\
  &CR is the thickness of concrete cove.(cm)\\
  &$k_a$ is the coefficient of active corrosion.(unitless)\\
  & The above equation prediction of remaining service life with respect to chloride attack.
  \\
  \hline
  Sources& James Clifton et al.~\cite{glassbr_spec} \\
  \hline
  Ref.\ By & -\\
  \hline
\end{tabular}
\end{minipage}\\
~\newline

%Instance Model 4

\noindent
\begin{minipage}{\textwidth}
\renewcommand*{\arraystretch}{1.5}
\begin{tabular}{| p{\colAwidth} | p{\colBwidth}|}
  \hline
  \rowcolor[gray]{0.9}
  Number & IM\refstepcounter{instnum}\theinstnum \\
  \hline  
  \label{weather}\\ % Move \label command to a separate line
  Label & \bf Prediction of remaining service life with weather data \\
  \hline
  Input & $A_\text{df}$, $A_\text{d}$ \\ % Add a closing $ here
  \hline
  Output & $t_r$ \\
  \hline
  Description &
  $t_r$ is the remaining service life of concrete. (year)\\
  & $A_{\text{df}}$ is the amount of damage at failure.(mm) \\
  & $A_{\text{d}}$ is the accumulated deterioration right now.(mm) \\
  & The above equation predicts the remaining service life with weather data. \\
  \hline
  Sources & James Clifton et al.~\cite{glassbr_spec} \\
  \hline
  Ref.\ By & - \\
  \hline
\end{tabular}
\end{minipage}\\
~\newline


%Instance Model 5

\noindent
\begin{minipage}{\textwidth}
\renewcommand*{\arraystretch}{1.5}
\begin{tabular}{| p{\colAwidth} | p{\colBwidth}|}
  \hline
  \rowcolor[gray]{0.9}
  Number& IM\refstepcounter{instnum}\theinstnum \label{tyf}\\
  \hline
  Label& \bf Predict the remaining service life based on the age at failure up to the time of inspection\\
  \hline
  Input& $t_f$, $t_i$ \\
  \hline
  Output & $t_r$\\
  \hline
  Description&
  $t_r$ is the remaining service life of concrete.(year)\\
  &$t_f$ is the average corrosion rate of a year.(year)\\
  &$t_i$ is the age of the concrete at the time of condition inspection.(year)\\
  & The above equation predict the remaining service life based on the age at failure up to the time of inspection.\\
  \hline
  Sources& James Clifton et al.~\cite{glassbr_spec} \\
  \hline
  Ref.\ By & -\\
  \hline
\end{tabular}
\end{minipage}\\
~\newline
\subsubsection{Input Data Constraints} \label{sec_DataConstraints}    

There are no constraints regarding input variables or specification parameter values.

\subsubsection{Properties of a Correct Solution} \label{sec_CorrectSolution}

There are no constraints regarding output variables.

\section{Requirements}

This section provides the functional requirements, the business tasks that the
software is expected to complete, and the nonfunctional requirements, the
qualities that the software is expected to exhibit.

\subsection{Functional Requirements}

\noindent \begin{itemize}

\item[R\refstepcounter{reqnum}\thereqnum \label{R_Inputs}:] Input the weather data and concrete degradation data.

\item[R\refstepcounter{reqnum}\thereqnum \label{R_Output}:] Output the prediction result.

\item[R\refstepcounter{reqnum}\thereqnum \label{R_Calculate}:] Utilize the input from the previous step to calculate the average corrosion rate and time to failure for the concrete.

\item[R\refstepcounter{reqnum}\thereqnum \label{R_VerifyOutput}:]Verify that the output result is generated from the input data.
\end{itemize}

\subsection{Nonfunctional Requirements}

\noindent \begin{itemize}

\item[NFR\refstepcounter{nfrnum}\thenfrnum \label{NFR_Usability}:]
  \textbf{Usability} With the UI, users are able to input data and generate output effortlessly.

\item[NFR\refstepcounter{nfrnum}\thenfrnum \label{NFR_Accuracy}:]
  \textbf{Accuracy} With the given input, the system is capable of generating accurate results.

\item[NFR\refstepcounter{nfrnum}\thenfrnum \label{NFR_Reusable}:] \textbf{Reusable} The code is modularized and flexible to extend functions.
\end{itemize}

\section{Likely Changes}    

\noindent 
\begin{itemize}

\item[LC\refstepcounter{lcnum}\thelcnum\label{LC_meaningfulLabel}:] The \hyperref[kdconstant]{Assumption \ref*{kdconstant}} restricts the category of degradations that might be experienced in the future.

\item[LC\refstepcounter{lcnum}\thelcnum\label{LC_meaningfulLabe2}:] The \hyperref[weather]{Assumption \ref*{weather}} restricts the weather factor to only be \(R_h\) that might be experienced in the future.
\end{itemize}
\section{Unlikely Changes}    

\noindent \begin{itemize}

\item[UL\refstepcounter{lcnum}\thelcnum\label{UL_meaningfulLabel}:] The \hyperref[Months]{Assumption \ref*{Months}} is fixed and unlikely to change.

\end{itemize}

\section{Traceability Matrices and Graphs}

The purpose of the traceability matrices is to provide easy references on what has to be additionally modified if a certain component is changed.  Every time a component is changed, the items in the column of that component that are marked with an ``X'' may have to be modified as well. Table~\ref{Table:trace} shows the dependencies of theoretical models, general definitions, data definitions, and instance models with each other. Table~\ref{Table:R_trace} shows the dependencies of instance models, requirements, and data constraints on each other. Table~\ref{Table:A_trace} shows the dependencies of theoretical models, general definitions, data definitions, instance models, and likely changes on the assumptions.

\begin{table}[htbp]
\centering
\begin{tabular}{|c|c|c|c|c|c|c|}
\hline
 & A1 & A2 &  A3& A4 & A5 & A6 \\
\hline
DD1 &  &  &  &  &  &  \\ \hline
DD2 &  &  &  & X & X & X \\ \hline
DD3 & X & X &  &  &  &  \\ \hline
DD4 &  &  &  &  &  &  \\ \hline
IM1 &  &  &  &  &  &  \\ \hline
IM2 &  &  &  &  &  &  \\ \hline
IM3 &  &  &  &  &  &  \\ \hline
IM4&  &  &  & X & X & X \\ \hline
IM5 &  &  &  &  &  &  \\ \hline
LC1 &  &  &  &  &  &  \\ \hline
LC2 &  &  &  &  &  &  \\ \hline
UL1 &  &  &  &  &  &  \\ \hline
\end{tabular}
\caption{Traceability Matrix Showing the Connections Between Assumptions and Other Items}
\label{Table:A_trace}
\end{table}
\newpage

\begin{table}[htbp]
\centering
\begin{tabular}{|c|c|c|c|c|c|c|c|c|c|c|c|c|}
\hline
 & DD1 & DD2& DD3 & DD4 & IM1 & IM2 & IM3& IM4 & IM5 & LC1& LC2 & UC1\\
\hline
DD1 &  &  &  &  &  &  &  &  &  &  &  &  \\ \hline
DD2 &  &  &  &  &  &  &  &  &  &  &  &  \\ \hline
DD3 &  &  &  &  &  &  &  &  &  &  &  &  \\ \hline
DD4 &  &  &  &  &  &  &  &  &  &  &  &  \\ \hline
IM1 &  &  &  &  &  &  &  &  &  &  &  &  \\ \hline
IM2 &  &  &  &  &  &  &  &  &  &  &  &  \\ \hline
IM3 &  &  &  &  &  &  &  &  &  &  &  &  \\ \hline
IM4 & X &  &  &  &  &  &  &  &  &  &  &   \\ \hline
IM5 &  &  &  &  &  &  &  &  &  &  &  &   \\ \hline
LC1 &  &  &  &  &  &  &  &  &  &  &  &  \\ \hline
LC2 &  &  &  &  &  &  &  &  &  &  &  &  \\ \hline
UC1&  &  &  &  &  &  &  &  &  &  &  &  \\ \hline
\end{tabular}
\caption{Your table caption}
\label{tab:your_table}
\end{table}
\newpage

\begin{table}[htbp]
\centering
\begin{tabular}{|c|c|c|c|c|c|c|c|c|}
\hline
 & IM1 & IM2 & IM3& IM4 & IM5 & R1 & R2 &R3 &R4 \\ \hline
IM1 &  &  &  &  &  &  &  &  \\ \hline
IM2 &  &  &  &  &  &  &  &  \\ \hline
IM3 &  &  &  &  &  &  &  &  \\ \hline
IM4 &  &  &  &  &  &  &  &  \\ \hline
IM5 &  &  &  &  &  &  &  &  \\ \hline
R1 &  &  &  &  &  &  &  &  \\ \hline
R2 &  &  &  &  &  &  &  &  \\ \hline
R3 &  &  &  &  &  &  &  &  \\ \hline
R4 &  &  &  &  &  &  &  &  \\ \hline
\end{tabular}
\caption{Traceability Matrix Showing the Connections Between Requirements and Instance Models}
\label{Table:R_trace}
\end{table}
\newpage

\begin{thebibliography}{9} 
    \bibitem{glassbr_spec}
    Nikitha Krithnan and W. Spencer Smith. \textit{Software Requirements Specification forGlassBR.}    \url{https://jacquescarette.github.io/Drasil/examples/glassbr/SRS/srs/GlassBR_SRS.html}

    \bibitem{srstem}
    Smith, W. Spencer.(2024). \textit{SRS}. GitHub. \url{https://github.com/smiths/capTemplate/blob/main/docs/SRS/SRS.pdf}
    
    \bibitem{historical_data}
    \textit{Historical Data.}Government of Canada. \url{https://climate.weather.gc.ca/historical_data/search_historic_data_e.html}
    
    \bibitem{concrete_life_1991}
    Clifton, J. R.(1991). 
    \textit{Methods for Predicting Remaining Life of Concrete in Structures.}
    
    \bibitem{concrete_life_1992}
    Clifton, J. R. and Pommersheim, J. M. (1992). 
    \textit{Methods for Predicting Remaining Life of Concrete in Structures.}   \url{https://nvlpubs.nist.gov/nistpubs/Legacy/IR/nistir4954.pdf}

    \bibitem{corrosionpedia}
    \textit{Corrosion Rate.}(2018, September 12) Corrosionpedia. \url{https://www.corrosionpedia.com/definition/337/corrosion-rate}

    \bibitem{propagation}
    Andrade, C.\textit{Propagation of reinforcement corrosion: principles, testing and modelling.}(2018) Corrosionpedia. \url{https://doi.org/10.1617/s11527-018-1301-1}    
\newpage
\paragraph{Appendix}
\begin{table}[htbp]
    \begin{tabular}{cc}
        Degradation Process & Value of n\\
        carbonation & 1/2\\
        diffusion of chloride ions & 1/2\\
        acid attack (siliceous aggregate) & 1/2\\
        acid attack (carbonate aggregate) & 1\\
        sulfate attack & 1\\
        frost attack & 1\\
        active corrosion of steel reinforcement (propagation) & 1\\
    \end{tabular}
    \caption{Values of n Obtained from Models}
    \label{tab:factors}
\end{table}

\end{document}