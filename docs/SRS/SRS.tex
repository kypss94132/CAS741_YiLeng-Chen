% THIS DOCUMENT IS TAILORED TO REQUIREMENTS FOR SCIENTIFIC COMPUTING.  IT SHOULDN'T
% BE USED FOR NON-SCIENTIFIC COMPUTING PROJECTS
\documentclass[12pt]{article}

\usepackage{amsmath, mathtools}
\usepackage{amsfonts}
\usepackage{amssymb}
\usepackage{graphicx}
\usepackage{colortbl}
\usepackage{xr}
\usepackage{hyperref}
\usepackage{longtable}
\usepackage{xfrac}
\usepackage{tabularx}
\usepackage{float}
\usepackage{siunitx}
\usepackage{booktabs}
\usepackage{caption}
\usepackage{pdflscape}
\usepackage{afterpage}

\usepackage[round]{natbib}

%\usepackage{refcheck}

\hypersetup{
    bookmarks=true,         % show bookmarks bar?
    colorlinks=true,       % false: boxed links; true: colored links
    linkcolor=red,          % color of internal links (change box color with linkbordercolor)
    citecolor=green,        % color of links to bibliography
    filecolor=magenta,      % color of file links
    urlcolor=cyan           % color of external links
}

%%% Comments

\usepackage{color}

\newif\ifcomments\commentstrue %displays comments
%\newif\ifcomments\commentsfalse %so that comments do not display

\ifcomments
\newcommand{\authornote}[3]{\textcolor{#1}{[#3 ---#2]}}
\newcommand{\todo}[1]{\textcolor{red}{[TODO: #1]}}
\else
\newcommand{\authornote}[3]{}
\newcommand{\todo}[1]{}
\fi

\newcommand{\wss}[1]{\authornote{blue}{SS}{#1}} 
\newcommand{\plt}[1]{\authornote{magenta}{TPLT}{#1}} %For explanation of the template
\newcommand{\an}[1]{\authornote{cyan}{Author}{#1}}

%%% Common Parts

\newcommand{\progname}{ProgName} % PUT YOUR PROGRAM NAME HERE
\newcommand{\authname}{Team \#, Team Name
\\ Student 1 name
\\ Student 2 name
\\ Student 3 name
\\ Student 4 name} % AUTHOR NAMES                  

\usepackage{hyperref}
    \hypersetup{colorlinks=true, linkcolor=blue, citecolor=blue, filecolor=blue,
                urlcolor=blue, unicode=false}
    \urlstyle{same}
                                


% For easy change of table widths
\newcommand{\colZwidth}{1.0\textwidth}
\newcommand{\colAwidth}{0.13\textwidth}
\newcommand{\colBwidth}{0.82\textwidth}
\newcommand{\colCwidth}{0.1\textwidth}
\newcommand{\colDwidth}{0.05\textwidth}
\newcommand{\colEwidth}{0.8\textwidth}
\newcommand{\colFwidth}{0.17\textwidth}
\newcommand{\colGwidth}{0.5\textwidth}
\newcommand{\colHwidth}{0.28\textwidth}

% Used so that cross-references have a meaningful prefix
\newcounter{defnum} %Definition Number
\newcommand{\dthedefnum}{GD\thedefnum}
\newcommand{\dref}[1]{GD\ref{#1}}
\newcounter{datadefnum} %Datadefinition Number
\newcommand{\ddthedatadefnum}{DD\thedatadefnum}
\newcommand{\ddref}[1]{DD\ref{#1}}
\newcounter{theorynum} %Theory Number
\newcommand{\tthetheorynum}{TM\thetheorynum}
\newcommand{\tref}[1]{TM\ref{#1}}
\newcounter{tablenum} %Table Number
\newcommand{\tbthetablenum}{TB\thetablenum}
\newcommand{\tbref}[1]{TB\ref{#1}}
\newcounter{assumpnum} %Assumption Number
\newcommand{\atheassumpnum}{A\theassumpnum}
\newcommand{\aref}[1]{A\ref{#1}}
\newcounter{goalnum} %Goal Number
\newcommand{\gthegoalnum}{GS\thegoalnum}
\newcommand{\gsref}[1]{GS\ref{#1}}
\newcounter{instnum} %Instance Number
\newcommand{\itheinstnum}{IM\theinstnum}
\newcommand{\iref}[1]{IM\ref{#1}}
\newcounter{reqnum} %Requirement Number
\newcommand{\rthereqnum}{R\thereqnum}
\newcommand{\rref}[1]{R\ref{#1}}
\newcounter{nfrnum} %NFR Number
\newcommand{\rthenfrnum}{NFR\thenfrnum}
\newcommand{\nfrref}[1]{NFR\ref{#1}}
\newcounter{lcnum} %Likely change number
\newcommand{\lthelcnum}{LC\thelcnum}
\newcommand{\lcref}[1]{LC\ref{#1}}

\usepackage{fullpage}

\newcommand{\deftheory}[9][Not Applicable]
{
\newpage
\noindent \rule{\textwidth}{0.5mm}

\paragraph{RefName: } \textbf{#2} \phantomsection 
\label{#2}

\paragraph{Label:} #3

\noindent \rule{\textwidth}{0.5mm}

\paragraph{Equation:}

#4

\paragraph{Description:}

#5

\paragraph{Notes:}

#6

\paragraph{Source:}

#7

\paragraph{Ref.\ By:}

#8

\paragraph{Preconditions for \hyperref[#2]{#2}:}
\label{#2_precond}

#9

\paragraph{Derivation for \hyperref[#2]{#2}:}
\label{#2_deriv}

#1

\noindent \rule{\textwidth}{0.5mm}

}

\begin{document}

\title{Software Requirements Specification for \progname: Concrete Remaining Life Prediction} 
\author{Yi-Leng Chen}
\date{February 5, 2024}
	
\maketitle
~\newpage

\pagenumbering{roman}

\tableofcontents

~\newpage

\section*{Revision History}

\begin{tabularx}{\textwidth}{p{3cm}p{2cm}X}
\toprule {\bf Date} & {\bf Version} & {\bf Notes}\\
\midrule
February 5 & 1.0 & Initial Documentation\\
\bottomrule
\end{tabularx}

~\newpage

\section{Reference Material}

This section records information for easy reference.

\subsection{Table of Units}

Throughout this document SI (Syst\`{e}me International d'Unit\'{e}s) is employed as the unit system.  In addition to the basic units, several derived units are used as described below.  For each unit, the symbol is given followed by a description of the unit and the SI name.
~\newline

\renewcommand{\arraystretch}{1.2}
%\begin{table}[ht]
  \noindent \begin{tabular}{l l l} 
    \toprule		
    \textbf{symbol} & \textbf{unit} & \textbf{SI}\\
    \midrule 
    \si{\metre} & length & metre\\
    \si{\second} & time & second\\
    \si{\celsius} & temperature & centigrade\\
    \si{\ampere} & electric current & ampere\\
    \si{\micro} & micro & \(10^{-6}\) \\
    \bottomrule
  \end{tabular}
  %	\caption{Provide a caption}
%\end{table}

\subsection{Table of Symbols}

The table that follows summarizes the symbols used in this document along with
their units.  The choice of symbols was made to be consistent with the heat
transfer literature and with existing documentation for solar water heating
systems.  The symbols are listed in alphabetical order.

\renewcommand{\arraystretch}{1.2}
%\noindent \begin{tabularx}{1.0\textwidth}{l l X}
\noindent \begin{longtable*}{l l p{12cm}} \toprule
\textbf{symbol} & \textbf{unit} & \textbf{description}\\
\midrule 
$A_\text{cr}$ & \si[per-mode=symbol]{{}\micro\ampere\per\centi\meter\squared}& Average corrosion rate of a year \\
$A_d$ & \si[per-mode=symbol] {\milli\meter} & Accumulated deterioration at time $t_y$ \\
$A_\text{df}$ & \si[per-mode=symbol] {\milli\meter} & Amount of damage at failure \\
$C_1$ & - & Coefficient weighing the impact of corrosion rates (Icorr) in $M_1$\\
$C_2$ & - & Coefficient weighing the impact of corrosion rates (Icorr) in $M_2$\\
$C_3$ & - & Coefficient weighing the impact of corrosion rates (Icorr) in $M_3$\\

$H_r$ & \% & Relative humidity\\
$I__\text{corr}$ & - & Corrosion Rates (Icorr)\\

$k_d$ & \si[per-mode=symbol]{{}\milli\meter\per\text{year}} & Factors influencing deterioration at time $t_y$ \\

$M_1$ & \SI{}{\text{month}} & Number of months that relative humidity is below 70\%\\
$M_2$ & \SI{}{\text{month}} & Number of months that relative humidity is between 70 and 100\% \\
$M_3$ & \SI{}{\text{month}} & Number of months that rain occurs\\

$n$ & year & Time order \\

$R_d$ & \si[per-mode=symbol]{{}\meter\per\second} & Rate of concrete degradation \\

$t_i$ & year & Age of the concrete at the time of condition inspection\\
$t_r$ & year & Remaining service life of concrete\\
$t_y$ & year & Service life of concrete\\
$t_f$ & year & Time to failure\\

\\ 
\bottomrule
\end{longtable*}

\subsection{Abbreviations and Acronyms}

\renewcommand{\arraystretch}{1.2}
\begin{tabular}{l l} 
  \toprule		
  \textbf{symbol} & \textbf{description}\\
  \midrule 
  A & Assumption\\
  DD & Data Definition\\
  GD & General Definition\\
  GS & Goal Statement\\
  Icorr & Corrosion rates\\
  IM & Instance Model\\
  LC & Likely Change\\
  PS & Physical System Description\\
  R & Requirement\\
  RH & Relative humidity \\
  SRS & Software Requirements Specification\\
  TM & Theoretical Model\\
  UI & User Interface \\
  \bottomrule
\end{tabular}\\


\pagenumbering{arabic}

\section{Introduction}
In response to the critical need for predicting the remaining life of concrete structures, this project aims to develop a program that implements theories introduced by a United States laboratory in 1992. Their work emphasized the importance of estimating remaining service life to enable property owners to plan for repairs or demolitions proactively. The program's goal is to predict the remaining service life of concrete structures, offering a valuable tool for effective maintenance and decision-making. \\
\\
The upcoming section will present a roadmap for the Software Requirements Specification (SRS) of the program. This segment elucidates the document's purpose, outlines the scope of the requirements, and describes the characteristics of the target audience for this document.

\subsection{Purpose of Document}

The primary aim of this document is to outline the requirements of the Concrete Remaining Life Prediction program. This encompasses background information, goals, assumptions, theoretical models, definitions, and other details related to model derivation. These components collectively enable the audience to gain a clear understanding and verify the purpose and scientific basis of this program.

\subsection{Scope of Requirements} 
This document explores the impact of climate data and the level of concrete deterioration on concrete structures, with the aim of implementing a predictive program.

\subsection{Characteristics of Intended Reader} \label{sec_IntendedReader}
The intended Reviewers of this documentation should possess an understanding of meteorology or building environments. Additionally, a basic grasp of high-school level mathematics and chemistry is recommended. For users of the prediction program can have a lower level of expertise, as further explained in the ``User Characteristics'' section (Section~\ref{SecUserCharacteristics}).

\subsection{Organization of Document}

The organization of this document follows the SRS template provided by Dr. Smith and the GlassBR SRS document.

\section{General System Description}

This section provides general information about the system.  It identifies the interfaces between the system and its environment, describes the user characteristics and lists the system constraints.  

\subsection{System Context}
Figure 1 illustrates the system context. In this representation, a circle signifies an external entity, which, in this case, is the user. The rectangle represents the prediction program, and arrows depict the data flow between the system and its environment.

\begin{figure}
    \centering
    \includegraphics[width=0.9\linewidth]{System Context.JPG}
    \caption{System Context}
    \label{fig:enter-label}
\end{figure}

The interaction between the product and the user is through a UI. The responsibilities of the user and the system are as follows: 

\begin{itemize}
\item User Responsibilities: Provide the input data related to climate and concrete deterioration level, ensuring that the entry data is valid.
\end{itemize}
\begin{itemize}
\item Prediction Program Responsibilities: Determine if the inputs satisfy the required program constraints and predict the remaining service life of concrete.
\end{itemize}

\subsection{User Characteristics} \label{SecUserCharacteristics}
\begin{itemize}
\item The end user is expected to possess basic computer literacy for effectively handling the software.
\item The end user is expected to have an understanding of the theory behind concrete degradation measurement and its root causes.
\end{itemize}

\subsection{System Constraints}

There are no system constraints.

\section{Specific System Description}

This section first presents the problem description, which gives a high-level view of the problem to be solved.  This is followed by the solution characteristics specification, which presents the assumptions, theories, definitions and finally
the instance models. 

\subsection{Problem Description} \label{Sec_pd}

A system is required to predict the remaining service life of concrete by integrating climate data and concrete degradation levels.

\subsubsection{Terminology and  Definitions}

This subsection provides a list of terms that are used in the subsequent sections and their meaning, with the purpose of reducing ambiguity and making it easier to correctly understand the requirements:

\begin{enumerate}
    \item Corrosion rates (Icorr): The flow of electric charge associated with the corrosion reactions occurring on a metal surface.

    \item Deterioration Factors:
    \begin{itemize}
        \item Carbonation
    \end{itemize}
    \begin{itemize}
        \item Diffusion of chloride ions
    \end{itemize}
    \begin{itemize}
        \item Acid attack (siliceous aggregate)
    \end{itemize}
    \begin{itemize}
        \item Acid attack (carbonate aggregate)
    \end{itemize}
    \begin{itemize}
        \item Sulfate attack
    \end{itemize}
    \begin{itemize}
        \item Frost attack
    \end{itemize}
    \begin{itemize}
        \item Active corrosion of steel
    \end{itemize}
    \begin{itemize}
        \item Reinforcement (propagation)
    \end{itemize}

    \item Rain: Precipitation in the form of liquid water droplets greater than 0.5 mm. If widely scattered, the drop size may be smaller.

    \item Relative Humidity: Relative humidity in percent (\%) is the ratio of the quantity of water vapour the air contains compared to the maximum amount it can hold at that particular temperature. 
\end{enumerate}

\subsubsection{Physical System Description} \label{sec_phySystDescrip}

The physical system regarding how deterioration occurs is not discussed in this study.

\subsubsection{Goal Statements}

Develop a prediction program that implements the theories, with the goal of predicting the remaining service life of concrete structures.

\subsection{Solution Characteristics Specification}

The instance models that govern this program are presented in
Subsection~\ref{sec_instance}.  The information to understand the meaning of the
instance models and their derivation is also presented, so that the instance
models can be verified.

\subsubsection{Assumptions} \label{sec_assumpt}

This section simplifies the original problem and helps in developing the theoretical model by filling in the missing information for the physical system. The numbers given in the square brackets refer to the theoretical model [TM],
general definition [GD], data definition [DD], instance model [IM], or likely change [LC], in which the respective assumption is used.

\begin{itemize}

A1: The values of $k_d$ remain constant throughout the entire deterioration period. \\
A2: The factors influencing the remaining life of concrete and causing degradation are restricted to the following: carbonation, diffusion of chloride ions, acid attack (siliceous aggregate), acid attack (carbonate aggregate), sulfate attack, frost attack, active corrosion of steel and reinforcement (propagation). \\
A3: Disregard self-conditions of concrete, such as cover thickness. \\
A4: Unless specifically mentioned, consider multiple degradation processes occurring simultaneously, making the time order n=1. \\
A5: If only one degradation process occurs, implicitly refer to Table 1 to determine the exact value of n.\\
A6: All information obtained from inspections is accurate. \\
A7: Relative humidity is the only factor that influences Icorr. \\
A8: Distribute based on $M_1$ , $M_2$ and $M_3$ three periods. \\
A9: The relative humidity value needs to remain consistently below 70\% for a month to be considered in the calculation of the M1 value. The same applies to the $M_2$ and $M_3$ values. \\
A10: The relationship between M1 and $M_2$ is mutually exclusive to $M_3$. In other words, there is no rain during the $M_1$ and $M_2$ periods.

\end{itemize}

\subsubsection{Theoretical Models}\label{sec_theoretical}

This section focuses on the general equations and laws that this program is based on.
~\newline

\noindent
\begin{minipage}{\textwidth}
\renewcommand*{\arraystretch}{1.5}
\begin{tabular}{| p{\colAwidth} | p{\colBwidth}|}
  \hline
  \rowcolor[gray]{0.9}
  Number& TM\refstepcounter{instnum}\theinstnum \label{ewat}\\
  \hline
  Label& \bf Remaining service life\\
  \hline
  Equation & $t_r$=$t_f$-$t_i$ \\
  \hline
  Description & $t_{yf}$ is time to failure. \\
              & $t_i$ is the age of the concrete at the time of condition inspection. \\
              & $t_r$ is the remaining service life of the concrete. \\
  & The above equation calculate the remaining service life.\\
  \hline
  Sources& - \\
  \hline
  Ref.\ By & -\\
  \hline
\end{tabular}
\end{minipage}\\

\noindent
\begin{minipage}{\textwidth}
\renewcommand*{\arraystretch}{1.5}
\begin{tabular}{| p{\colAwidth} | p{\colBwidth}|}
  \hline
  \rowcolor[gray]{0.9}
  Number& TM\refstepcounter{instnum}\theinstnum \label{ewat}\\
  \hline
  Label& \bf Average corrosion rate of a year\\
 \hline
  Equation & 
\begin{equation*}
A_{cr} = \frac{C_1 \cdot M_1 + C_2 \cdot M_2 + C_3 \cdot M_3}{12}
\end{equation*} \\
\hline
Description &
$C_1$ is Coefficient weighing the impact of corrosion rates (Icorr) in $M_1$. 
$C_2$ is Coefficient weighing the impact of corrosion rates (Icorr) in $M_2$. 
$C_3$ is Coefficient weighing the impact of corrosion rates (Icorr) in $M_3$.  
$M_1$ is Number of months that relative humidity is below 70\%.  
$M_2$ is Number of months that relative humidity is between 70 and 100\%.  
$M_3$ is Number of months that rain occurs.  
The above equation calculates average corrosion rate of a year.  \\
\hline
  Sources& - \\
  \hline
  Ref.\ By & -\\
  \hline
\end{tabular}
\end{minipage}\\

\subsubsection{General Definitions}\label{sec_gendef}

There are no general definitions.

\subsubsection{Data Definitions}\label{sec_datadef}

This section collects and defines all the data needed to build the instance models. The dimension of each quantity is also given.  
~\newline

\noindent
\begin{minipage}{\textwidth}

\renewcommand*{\arraystretch}{1.5}
\begin{tabular}{| p{\colAwidth} | p{\colBwidth}|}
\hline
\rowcolor[gray]{0.9}
Number& DD\refstepcounter{datadefnum}\thedatadefnum \label{FluxCoil}\\
\hline
Label& \bf Month data\\
\hline
Symbol & M\\
\hline
SI Units & months\\
\hline
Equation & $M=M_1+M_2+M_3$\\
\hline
Description & 
M represents the weather data for 12 months. Specifically, M1 denotes the number of months with relative humidity below 70\%, M2 indicates the number of months with relative humidity between 70 and 100\%, and M3 represents the number of months with rainfall.\\
\hline
  Sources& Methods for Predicting Remaining Life of Concrete in Structures \\
  \hline
  Ref.\ By & TM2\\
  \hline
\end{tabular}
\end{minipage}\\

~\newline

\noindent
\begin{minipage}{\textwidth}

\renewcommand*{\arraystretch}{1.5}
\begin{tabular}{| p{\colAwidth} | p{\colBwidth}|}
\hline
\rowcolor[gray]{0.9}
Number& DD\refstepcounter{datadefnum}\thedatadefnum \label{FluxCoil}\\
\hline
Label & Coefficient weighing the impact of corrosion rates (Icorr) in $M$ \\
\hline
Symbol & $M_1$, $M_2$, $M_3$, $C_1$, $C_2$, $C_3$ \\
\hline
SI Units & - \\
\hline
Equation & 
\begin{cases}
    M_1, C_1 = 0.1 \\
    M_2, C_2 = 1.0 \\
    M_3, C_3 = 10 \\
\end{cases} \\
\hline
Description & 
The coefficient for $M_1$ will be 0.1, for $M_2$ it will be 1.0, and for $M_3$ it will be 10.\\
\hline
  Sources& Methods for Predicting Remaining Life of Concrete in Structures \\
  \hline
  Ref.\ By & TM2\\
  \hline
\end{tabular}
\end{minipage}\\

\subsubsection{Instance Models} \label{sec_instance}    

This section transforms the problem defined in Section~\ref{Sec_pd} into 
one which is expressed in mathematical terms. It uses concrete symbols defined 
in Section~\ref{sec_datadef} to replace the abstract symbols in the models 
identified in Sections~\ref{sec_theoretical} and~\ref{sec_gendef}.

~\newline

%Instance Model 1

\noindent
\begin{minipage}{\textwidth}
\renewcommand*{\arraystretch}{1.5}
\begin{tabular}{| p{\colAwidth} | p{\colBwidth}|}
  \hline
  \rowcolor[gray]{0.9}
  Number& IM1\\
  \hline
  Label& \bf Amount of concrete degradation\\
  \hline
  Input& $k_d$, $t_y$, n\\
  \hline
  Output& $A_d$\\
 %\iref{epcm} \\
  \hline
  Description 
  &$A_d$ is Accumulated deterioration at time $t_y$. \\
  &$k_d$ is Factors influencing deterioration. \\
  &$t_y$ is Service life of concrete. \\
  & n is time order \\
  & The above equation calculate the amount of concrete degradation.\\
  \hline
  Sources& - \\
  \hline
  Ref.\ By & TM1\\
  \hline
\end{tabular}
\end{minipage}\\
~\newline

\noindent
\begin{minipage}{\textwidth}
\renewcommand*{\arraystretch}{1.5}
\begin{tabular}{| p{\colAwidth} | p{\colBwidth}|}
  \hline
  \rowcolor[gray]{0.9}
  Number& IM2\\
  \hline
  Label& \bf Rate of concrete degradation\\
  \hline
  Input& n, $k_d$, $t_y$, n-1\\
  \hline
  Output& $R_d$\\
 %\iref{epcm} \\
  \hline
  Description 
  &$R_d$ is rate of degradation. \\
  &$k_d$ is factors influencing deterioration. \\
  &$t_y$ is service life of concrete. \\
  & n is time order \\
  & The above equation calculate the rate of degradation.\\
  \hline
  Sources& - \\
  \hline
  Ref.\ By & TM1\\
  \hline
\end{tabular}
\end{minipage}\\
~\newline

\noindent
\begin{minipage}{\textwidth}
\renewcommand*{\arraystretch}{1.5}
\begin{tabular}{| p{\colAwidth} | p{\colBwidth}|}
  \hline
  \rowcolor[gray]{0.9}
  Number& IM3\\
  \hline
  Label& \bf Calculate the time to get failure\\
  \hline
  Input & $A_{df}$, $k_d$, $1/n$ \\
  \hline
  Output & $t_{yf}$ \\
  \hline
  Description & $A_{df}$ is the amount of damage at failure. \\
              & $k_d$ is the factor influencing deterioration. \\
              & $t_{yf}$ is the time to failure. \\
  \hline
  & n is time order \\
  & The above equation calculate the time to get failure.\\
  \hline
  Sources& - \\
  \hline
  Ref.\ By & TM1\\
  \hline
\end{tabular}
\end{minipage}\\
~\newline

\subsubsection{Input Data Constraints} \label{sec_DataConstraints}    

Table~\ref{TblInputVar} shows the data constraints on the input output
variables.  The column for physical constraints gives the physical limitations
on the range of values that can be taken by the variable.  The column for
software constraints restricts the range of inputs to reasonable values.  The
software constraints will be helpful in the design stage for picking suitable
algorithms.  The constraints are conservative, to give the user of the model the
flexibility to experiment with unusual situations.  The column of typical values
is intended to provide a feel for a common scenario.  The uncertainty column
provides an estimate of the confidence with which the physical quantities can be
measured.  This information would be part of the input if one were performing an
uncertainty quantification exercise.

\begin{table}[!h]
  \caption{Input Variables} \label{TblInputVar}
  \renewcommand{\arraystretch}{1.2}
\noindent \begin{longtable*}{l l l l c} 
  \toprule
  \textbf{Var} & \textbf{Physical Constraints} & \textbf{Software Constraints} &
                             \textbf{Typical Value} & \textbf{Uncertainty}\\
  \midrule 
  $t_y$ & $t_y \ge 0$ & $-$ & 1 year & 10\% \\
  $n$ & $n \ge 0$ & $-$ & 1 & 10\% \\
  $A_{df}$ & $A_{df} \ge 0$ & $-$ & 10mm & 10\%
  $M_1$ & $12 \ge M_1 \ge 0$ & $-$ & 4months & 10\% \\
  $M_2$ & $12 \ge M_1 \ge 0$ & $-$ & 4months & 10\% \\
  $M_3$ & $12 \ge M_1 \ge 0$ & $-$ & 4months & 10\% \\
  \\
  \bottomrule
\end{longtable*}
\end{table}

\subsubsection{Properties of a Correct Solution} \label{sec_CorrectSolution}

\noindent
A correct solution must exhibit \plt{fill in the details}.  \plt{These
  properties are in addition to the stated requirements.  There is no need to
  repeat the requirements here.  These additional properties may not exist for
  every problem.  Examples include conservation laws (like conservation of
  energy or mass) and known constraints on outputs, which are usually summarized
  in tabular form.  A sample table is shown in Table~\ref{TblOutputVar}}

\begin{table}[!h]
\caption{Output Variables} \label{TblOutputVar}
\renewcommand{\arraystretch}{1.2}
\noindent \begin{longtable*}{l l} 
  \toprule
  \textbf{Var} & \textbf{Physical Constraints} \\
  \midrule 
  $A_d$ & $A_d \ge 0$\\
  $R_d$ & $R_d \ge 0$\\
  \\
  \bottomrule
\end{longtable*}
\end{table}

\section{Requirements}

This section provides the functional requirements, the business tasks that the
software is expected to complete, and the nonfunctional requirements, the
qualities that the software is expected to exhibit.

\subsection{Functional Requirements}

\noindent \begin{itemize}

\item[R\refstepcounter{reqnum}\thereqnum \label{R_Inputs}:] Input the weather data and concrete degradation data.

\item[R\refstepcounter{reqnum}\thereqnum \label{R_OutputInputs}:] Utilize the input from the previous step to calculate the average corrosion rate and time to failure for the concrete.

\item[R\refstepcounter{reqnum}\thereqnum \label{R_Calculate}:] Generated prediction result.

\end{itemize}

\subsection{Nonfunctional Requirements}

\noindent \begin{itemize}

\item[NFR\refstepcounter{nfrnum}\thenfrnum \label{NFR_Accuracy}:]
  \textbf{Accuracy} With the given input, the system is capable of generating accurate results.

\item[NFR\refstepcounter{nfrnum}\thenfrnum \label{NFR_Verifiable}:] \textbf{Reusable} The code is modularized.

\end{itemize}

\section{Likely Changes}    

\noindent \begin{itemize}

\item[LC\refstepcounter{lcnum}\thelcnum\label{LC_meaningfulLabel}:] IM1 is the likely to changes.

\end{itemize}

\section{Unlikely Changes}    

\noindent \begin{itemize}

\item[LC\refstepcounter{lcnum}\thelcnum\label{LC_meaningfulLabel}:] IM4 is fixed and unlikely to change.

\end{itemize}

\section{Traceability Matrices and Graphs}

The purpose of the traceability matrices is to provide easy references on what has to be additionally modified if a certain component is changed.  Every time a component is changed, the items in the column of that component that are marked with an ``X'' may have to be modified as well.  Table~\ref{Table:trace} shows the dependencies of theoretical models, general definitions, data definitions, and instance models with each other. Table~\ref{Table:R_trace} shows the dependencies of instance models, requirements, and data constraints on each other. Table~\ref{Table:A_trace} shows the dependencies of theoretical models, general definitions, data definitions, instance models, and likely changes on the assumptions.

\begin{table}[htbp]
\centering
\begin{tabular}{|c|c|c|c|c|c|c|c|}
\hline        
    & TM1 & TM2 & DD1 & DD2 & IM1 & IM2 & IM3 \\
\hline
TM1 & & & & & & & \\ \hline
TM2 & & & & & & & \\ \hline
DD1 & & X & & & & & \\ \hline
DD2 & & X & & & & & \\ \hline
IM1 & & & & & & & \\ \hline
IM2 & & & & & & & \\ \hline
IM3 & & & & & & & \\ \hline
\hline
\end{tabular}
\caption{Traceability Matrix Showing the Connections Between Items of Different Sections}
\label{Table:trace}
\end{table}
\section{Values of Auxiliary Constants}
\begin{table}[htbp]
    \centering
    \begin{tabular}{cccc}
    \hline        
    Symbol & Description & Value & Unit  \\
    \hline
    $A_d$ & Accumulated deterioration at time $t_y$ & 24 & mm \\
    $t_y$ & Time to failure  & 36 & years \\
    n & Time order  & 1 & - \\
    $k_d$ & Factors influencing deterioration  & 4 & mm/years \\
    $t_f$ & Time to failure & 144 & years \\
    \end{tabular}
    \caption{Caption}
    \label{tab:my_label}
\end{table}

\newpage

\begin{thebibliography}{9} % Use a number that's larger than the number of references
\bibitem{glassbr_spec}
Nikitha Krithnan and W. Spencer Smith. \textit{Software Requirements Specification for GlassBR.}
\url{https://jacquescarette.github.io/Drasil/examples/glassbr/SRS/srs/GlassBR_SRS.html}

\bibitem{glassbr_spec}
\textit{Historical Data.} (n.d.). Government of Canada. \url{https://climate.weather.gc.ca/historical_data/search_historic_data_e.html}

\bibitem{glassbr_spec}
Clifton, J. R., & Pommersheim, J. M. (1992). \textit{Methods for Predicting Remaining Life of Concrete in Structures.}
\url{https://nvlpubs.nist.gov/nistpubs/Legacy/IR/nistir4954.pdf}

\end{thebibliography}

\end{document}